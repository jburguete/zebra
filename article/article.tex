\documentclass[review,authoryear]{elsarticle}

\usepackage[utf8]{inputenc}
\usepackage[english]{babel}

\title{A coupled model of transport and decay of solutes and mussel larvae in
	pressurized networks}

\author[1]{Javier Burguete}
\ead{jburguete@eead.csic.es}

\author[1]{Borja Latorre}
\ead{borja.latorre@csic.es}

\author[1]{Pilar Paniagua}
\ead{pilucap@eead.csic.es}

\author[1]{Eva Medina}
\ead{emedina@eead.csic.es}

\author[1]{Javier Fernández-Pato}
\ead{jfernandez@eead.csic.es}

\author[1]{Enrique Playán}
\ead{enrique.playan@csic.es}

\author[1]{Nery Zapata}
\ead{v.zapata@csic.es}

\affiliation[1]{organization={Soil and Water, EEAD/CSIC},
	addressline={Av. Montañana 1005},
	postcode={50059},
	city={Zaragoza},
	country={Spain}}

\date{\today}

\newcommand{\CO}[1]{\left[#1\right]}
\newcommand{\EQ}[2]{\begin{equation}#1\label{#2}\end{equation}}
\newcommand{\FIG}[3]
{
	\begin{figure}[ht!]
		\centering
		\includegraphics[width=\textwidth]{#1}
		\caption{#2.\label{#3}}
	\end{figure}
}
\newcommand{\FIGII}[4]
{
	\begin{figure}[ht!]
		\centering
		\includegraphics[width=0.49\textwidth]{#1}
		\includegraphics[width=0.49\textwidth]{#2}
		\caption{#3.\label{#4}}
	\end{figure}
}
\newcommand{\FIGIIB}[6]
{
	\begin{figure}[ht!]
		\centering
		\begin{tabular}{cc}
			#2&#4\\
			\includegraphics[width=0.49\textwidth]{#1}&
			\includegraphics[width=0.49\textwidth]{#3}\\
		\end{tabular}
		\caption{#5.\label{#6}}
	\end{figure}
}
\newcommand{\FIGIIC}[5]
{
	\begin{figure}[ht!]
		\centering
		\begin{tabular}{cc}
			#2&#4\\
			&#5\\
			\includegraphics[width=0.45\textwidth]{#1}&
			\includegraphics[width=0.45\textwidth]{#3}
		\end{tabular}
	\end{figure}
}
\newcommand{\FIGIII}[5]
{
	\begin{figure}[ht!]
		\centering
		\includegraphics[width=0.49\textwidth]{#1}
		\includegraphics[width=0.49\textwidth]{#2}
		\includegraphics[width=0.49\textwidth]{#3}
		\caption{#4.\label{#5}}
	\end{figure}
}
\newcommand{\FIGIIIB}[8]
{
	\begin{figure}[ht!]
		\centering
		\begin{tabular}{cc}
			#2&#4\\
			\includegraphics[width=0.49\textwidth]{#1}&
			\includegraphics[width=0.49\textwidth]{#3}\\
		\end{tabular}
		#6\\
		\includegraphics[width=0.49\textwidth]{#5}
		\caption{#7.\label{#8}}
	\end{figure}
}
\newcommand{\FIGIV}[6]
{
	\begin{figure}[ht!]
		\centering
		\includegraphics[width=0.49\textwidth]{#1}
		\includegraphics[width=0.49\textwidth]{#2}
		\includegraphics[width=0.49\textwidth]{#3}
		\includegraphics[width=0.49\textwidth]{#4}
		\caption{#5.\label{#6}}
	\end{figure}
}
\newcommand{\FIGIVB}[9]
{
	\begin{figure}[ht!]
		\centering
		\begin{tabular}{cc}
			#2&#4\\
			\includegraphics[width=0.49\textwidth]{#1}&
			\includegraphics[width=0.49\textwidth]{#3}\\
			#6&#8\\
			\includegraphics[width=0.49\textwidth]{#5}&
			\includegraphics[width=0.49\textwidth]{#7}
		\end{tabular}
		\caption{#9.}
	\end{figure}
}
\newcommand{\FIGV}[7]
{
	\begin{figure}[ht!]
		\centering
		\includegraphics[width=0.49\textwidth]{#1}
		\includegraphics[width=0.49\textwidth]{#2}
		\includegraphics[width=0.49\textwidth]{#3}
		\includegraphics[width=0.49\textwidth]{#4}
		\includegraphics[width=0.49\textwidth]{#5}
		\caption{#6.\label{#7}}
	\end{figure}
}
\newcommand{\LL}[1]{\left\{#1\right\}}
\newcommand{\PA}[1]{\left(#1\right)}
\newcommand{\PARTIAL}[2]{\frac{\partial#1}{\partial#2}}

\begin{document}

\begin{abstract}
In this work we present ...
\end{abstract}

\begin{highlights}
\item 1st highlight
\end{highlights}

\begin{keyword}
mussel larvae, solute transport, solute decay, pressurized networks, irrigation
\end{keyword}

\maketitle

\section*{Notation}

\begin{tabular}{lcl}
$\Delta$ & = & temporal differences,\\
$\delta$ & = & spatial differences,\\
$\Psi^\pm$ & = & upwind flux limiter,\\
$\psi$ & = & flux limiter function,\\
$\theta$ & = & implicit parameter,\\
$A$ & = & cross sectional area,\\
$C$ & = & mussel larvae sources,\\
$c$ & = & mussel larvae concentration,\\
$f$ & = & friction factor of Darcy-Weissbach,\\
$K_b$ & = & base solute decay coefficient,\\
$K_w$ & = & wall solute decay coefficient,\\
$K_x$ & = & longitudinal dispersion coefficient,\\
$P$ & = & perimeter,\\
$Q$ & = & discharge,\\
$S$ & = & solute sources,\\
$s$ & = & solute concentration,\\
$t$ & = & time,\\
$v$ & = & velocity,\\
$x$ & = & longitudinal coordinate,\\
\end{tabular}

\section{Introduction}

\section{Theorical models}

\subsection{Model of transport and dispersion of solutes in a pipe}

The equation governing a one-dimensional transport and dispersion of solutes in
a pipe is the advection-diffusion Fick's law with sources due to external
injections or reaction decays:
\EQ{\PARTIAL{}{t}(A\,s)+\PARTIAL{}{x}(Q\,s)
	=S+\PARTIAL{}{x}\PA{K_x\,A\,\PARTIAL{s}{x}}}{EqTransportSolute}
where $t$ is the time, $A$ the cross sectional area of the pipe, $s$ the solute
concentration, $x$ the longitudinal coordinate, $Q$ the discharge, $S$ the
solute source and $K_x$ the longitudinal dispersion coefficient.

In this work, the following model, according to the \citet{Rutherford94} works,
for $K_x$ coefficient is used:
\EQ{K_x=10\,P\,|v|\,\sqrt{\frac{f}{2}}}{EqKx}  
with $P$ the perimeter of the pipe, $v$ the water velocity and $f$ friction
factor of Darcy-Weissbach \citep{Darcy58}.

\subsection{Model of solute decay in a pipe}

In the solute decay in a pipe two effects are taked into accout: the reaction of
the solute with the matter disolved in the water and the reaction with the pipe
walls. The following equation is used \citep{DiGianoZhang05}:
\EQ{\PARTIAL{c}{t}=-\PA{K_b-K_w\,\frac{P}{A}}\,c}{EqSoluteDecay}
with $K_b$ and $K_w$ the base and wall decay constants respectively. 

\subsection{Model of transport and dispersion of mussel larvae in a pipe}

The swimming capacity of the zebra mussel larvae can be considered negligible
compared to the water velocity in pressurized networks. Then, equation
(\ref{EqTransportSolute}), adapted to mussel larvae, is used for transport and
dispersion in pipes:
\EQ{\PARTIAL{}{t}(A\,c)+\PARTIAL{}{x}(Q\,c)
	=C+\PARTIAL{}{x}\PA{K_x\,A\,\PARTIAL{c}{x}}}{EqTransportLarvae}
with $c$ the larvae concentration and $C$ the sources of larvae.

\section{The numerical model}

\subsection{Model of advection-diffusion equation}

The Roe-Sweby \citep{Sweby84} second order in space and time TVD (total
variation diminishing) scheme is used to solve the advection term combined with
a centred implicit scheme of second order in space to solve the diffusion term
of the advection-diffusion equations (\ref{EqTransportSolute}) and
(\ref{EqTransportLarvae}):
\[\Delta s^n=s^{n+1}-s^n,\quad\delta s_{i+\frac12}=s_{i+1}-s_i,\quad
	\delta F_{i+\frac12}^n=\PA{1-v\,\frac{\Delta t}{\delta x}}\,v\,
	\delta s_{i+\frac12}^n,\]
\[
	\delta F^\pm=\frac{v\pm|v|}{2}\,\delta s_{i+\frac12},\quad
	\Psi_{i+\frac12}^\pm=\psi\PA{\frac{\delta F_{i+\frac12\pm1}^\pm}
	{\delta F_{i+\frac12}^\pm}},\]
\[\psi(r)=\max\CO{0,\;\min\PA{2\,r,\;\frac{1+r}{2},\;2}},\]
\[-\theta\,K_x\,\frac{\Delta t}{\delta x}\,\Delta s_{i-1}^n
	+\PA{\delta x+2\,\theta\,K_x\,\frac{\Delta t}{\delta x}}\,\Delta s_i^n
	-\theta\,K_x\,\frac{\Delta t}{\delta x}\,\Delta s_{i+1}^n\]
\[=(1-\theta)\,K_x\,\frac{\Delta t}{\delta x}\,\PA{\delta s_{i+\frac12}^n
	-\delta s_{i-\frac12}^n}\]
\[+\Delta t\,\left\{\frac12\,\PA{\Psi^+\,\delta F^+}_{i-\frac32}^n
	-\CO{\PA{1+\frac12\,\Psi^+}\,\delta F^+}_{i-\frac12}^n\right.\]
\EQ{
	\left.-\CO{\PA{1+\frac12\,\Psi^-}\,\delta F^-}_{i+\frac12}^n
	+\frac12\,\PA{\Psi^-\,\delta F^-}_{i+\frac32}^n\right\}}
	{EqAdvectionDiffusion}
with $\theta\in[0,1]$ the implicit parameter, $\psi$ the monotonized central
flux limiter function \citep{vanLeer77}, $\Delta$ is used for temporal
differences and $\delta$ for spatial differences. This scheme produces a
tridiagonal system of equations in $\Delta s_i^n$ that is solved by Gaussian
elimination.

For $\theta=\frac12$ the diffusion scheme is second order in time. The following
value is used for compromise in accuracy holding the TVD property:
\EQ{\theta=\max\PA{\frac12,\;1-\frac{\delta x^2}{2\,K_x\,\Delta t}}}{EqTheta}

\subsection{Model of mussel larvae cling in a pipe}

\subsection{Model of mussel larvae death}

\subsection{Model of junctions}

To solve the solute or larvae concentration at the network junctions, equality
of concentrations and mass conservation at all nodes involved in the junction is
imposed:
\EQ{s=\frac{\sum_jA_j\,s_j\,\delta x_j}{\sum_jA_j\,\delta x_j}}{EqJunction}
with $s$ the concentration at all the nodes and $j$ for each node.

\subsection{Model of inlets}

\section{Applications cases}

\subsection{Efecto de la variabilidad en la entrada}

Hipótesis:
\begin{itemize}
\item Red de la Comunidad de Regantes de Montesusín.
\item El número total de larvas es idéntico en cada modelo.
\item Tiempo medio de pegado $\approx$ 4 días.
\item Velocidad máxima de pegado 1.2 m/s.
\item No muerte.
\end{itemize}

\FIGIV{2021-mussel-input-constant.eps}{2021-mussel-input-measured.eps}
{2021-mussel-input-uniform.eps}{2021-mussel-input-random.eps}
{Entradas de larvas en función del tiempo. Año 2021}{Fig2021MusselInput}

\FIGIVB{2021-mussel-constant.pdf}{$+0\%$, R=$1$}
{2021-mussel-measured.pdf}{$-41.7\%$, R=$0.977$}
{2021-mussel-uniform.pdf}{$-28.4\%$, R=$0.976$}
{2021-mussel-random.pdf}{$-9.8\%$, R=$0.990$}
{Infestación con entrada de larvas constante en una ventana de tiempo
(arriba-izquierda), medida (arriba-derecha), uniforme (abajo-izquierda) y
aleatoria (abajo-derecha). Año 2021\label{Fig2021MusselRandom}}

\FIGIV{2022-mussel-input-constant.eps}
{2022-mussel-input-measured.eps}
{2022-mussel-input-uniform.eps}
{2022-mussel-input-random.eps}
{Entradas de larvas en función del tiempo. Año 2022}{Fig2022MusselInput}

\FIGIVB{2022-mussel-constant.pdf}{$+0\%$, R=$1$}
{2022-mussel-measured.pdf}{$+24.8\%$, R=$0.976$}
{2022-mussel-uniform.pdf}{$-21.8\%$, R=$0.957$}
{2022-mussel-random.pdf}{$-28.3\%$, R=$0.988$}
{Infestación con entrada de larvas constante en una ventana de tiempo
(arriba-izquierda), medida (arriba-derecha), uniforme (abajo-izquierda) y
aleatoria (abajo-derecha). Año 2022\label{Fig2022MusselRandom}}

\clearpage
\subsection{Efecto de la velocidad máxima de pegado}

Hipótesis:
\begin{itemize}
\item Red de la Comunidad de Regantes de Montesusín, año 2021.
\item Entrada de larvas constante.
\item Tiempo medio de pegado $\approx$ 4 días.
\item No muerte.
\end{itemize}

\FIGIIIB{2021-mussel-velocity-1.0.pdf}{$-9.1\%$, R=$0.971$}
{2021-mussel-constant.pdf}{$+0\%$, R=$1$}
{2021-mussel-velocity-2.0.pdf}{$+9.7\%$, R=$0.951$}
{Infestación con velocidad máxima de pegado de
1.0~m/s (arriba-izquierda), 1.2~m/s (arriba-derecha) y 2.0~m/s (abajo). Año
2021}{Fig2021MusselVelocity}

\FIGIIIB{2022-mussel-velocity-1.0.pdf}{$-2.8\%$, R=$0.995$}
{2022-mussel-constant.pdf}{$+0\%$, R=$1$}
{2022-mussel-velocity-2.0.pdf}{$+2.9\%$, R=$0.989$}
{Infestación con velocidad máxima de pegado de
1.0~m/s (arriba-izquierda), 1.2~m/s (arriba-derecha) y 2.0~m/s (abajo). Año
2022}{Fig2022MusselVelocity}

\clearpage
\subsection{Efecto de la probabilidad de pegado}

Hipótesis:
\begin{itemize}
\item Red de la Comunidad de Regantes de Montesusín.
\item Entrada de larvas constante.
\item Velocidad máxima de pegado 1.2 m/s.
\item No muerte.
\end{itemize}

\FIGIIIB{2021-mussel-cling-1.1.pdf}{$-44.6\%$, R=$0.986$}
{2021-mussel-constant.pdf}{$+0\%$, R=$1$}
{2021-mussel-cling-8.0.pdf}{$+273.3\%$, R=$0.961$}
{Infestación con tiempo medio de pegado de
14 días (arriba-izquierda), 4 días (arriba-derecha) y 1 día (abajo). Año 2021}
{Fig2021MusselCling}

\FIGIIIB{2022-mussel-cling-1.1.pdf}{$-41.1\%$, R=$0.995$}
{2022-mussel-constant.pdf}{$+0\%$, R=$1$}
{2022-mussel-cling-8.0.pdf}{$+171.6\%$, R=$0.955$}
{Infestación con tiempo medio de pegado de
14 días (arriba-izquierda), 4 días (arriba-derecha) y 1 día (abajo). Año 2022}
{Fig2022MusselCling}

\clearpage
\subsection{Efecto de la probabilidad de muerte}

Hipótesis:
\begin{itemize}
\item Red de la Comunidad de Regantes de Montesusín.
\item Entrada de larvas constante.
\item Tiempo medio de pegado $\approx$ 4 días.
\item Velocidad máxima de pegado 1.2 m/s.
\item Tasa de muerte de larvas: a los 14 días sobrevive el 10\%.
\item Tasa de muerte de adultos: a los 4 años sobrevive el 10\%.
\end{itemize}

\FIGIIB{2021-mussel-constant.pdf}{$+0\%$, R=$1$}
{2021-mussel-death.pdf}{$-9.8\%$, R=$0.997$}
{Infestación sin muerte
de larvas (izquierda) y con muerte (derecha). Año 2021}{Fig2021MusselDeath}

\FIGIIB{2022-mussel-constant.pdf}{$+0\%$, R=$1$}
{2022-mussel-death.pdf}{$-16.4\%$, R=$0.999$}
{Infestación sin muerte
de larvas (izquierda) y con muerte (derecha). Año 2022}{Fig2022MusselDeath}

\clearpage
\section{Efectos debidos a los venenos}

\subsection{Propagación de cloro en la red de Montesusín}

Hipótesis:
\begin{itemize}
\item Red de la Comunidad de Regantes de Montesusín.
\item Tasa de decaimiento del cloro: 0.36 día$^{-1}$.
\end{itemize}

\FIGIV{24-0.36-2021-chlorine-constant-0.eps}{24-0.36-2021-chlorine-constant-2.eps}
{24-0.36-2021-chlorine-constant-6665.eps}{24-0.36-2021-chlorine-constant-172.eps}
{Concentración de cloro en distintos nodos con entrada constante. Año 2021}
{Fig0362021ChlorineConstant}

\FIGIV{24-0.36-2022-chlorine-constant-0.eps}{24-0.36-2022-chlorine-constant-2.eps}
{24-0.36-2022-chlorine-constant-68.eps}{24-0.36-2022-chlorine-constant-172.eps}
{Concentración de cloro en distintos nodos con entrada constante. Año 2022}
{Fig0362022ChlorineConstant}

\FIGV{24-0.36-2021-chlorine-day-0.eps}{24-0.36-2021-chlorine-day-2.eps}
{24-0.36-2021-chlorine-day-65.eps}{24-0.36-2021-chlorine-day-6665.eps}
{24-0.36-2021-chlorine-day-172.eps}
{Concentración de cloro en distintos nodos con entrada con una ventana horaria
diaria. Año 2021}{Fig0362021ChlorineDay}

\FIGV{24-0.36-2022-chlorine-day-0.eps}{24-0.36-2022-chlorine-day-2.eps}
{24-0.36-2022-chlorine-day-6634.eps}{24-0.36-2022-chlorine-day-68.eps}
{24-0.36-2022-chlorine-day-172.eps}
{Concentración de cloro en distintos nodos con entrada con una ventana horaria
diaria. Año 2022}{Fig0362022ChlorineDay}

\FIGV{24-0.36-2021-chlorine-week-0.eps}{24-0.36-2021-chlorine-week-2.eps}
{24-0.36-2021-chlorine-week-68.eps}{24-0.36-2021-chlorine-week-123.eps}
{24-0.36-2021-chlorine-week-172.eps}
{Concentración de cloro en distintos nodos con entrada con una ventana horaria
semanal. Año 2021}{Fig0362021ChlorineWeek}

\FIGV{24-0.36-2022-chlorine-week-0.eps}{24-0.36-2022-chlorine-week-2.eps}
{24-0.36-2022-chlorine-week-6634.eps}{24-0.36-2022-chlorine-week-139.eps}
{24-0.36-2022-chlorine-week-172.eps}
{Concentración de cloro en distintos nodos con entrada con una ventana horaria
semanal. Año 2022}{Fig0362022ChlorineWeek}

\clearpage
Hipótesis:
\begin{itemize}
\item Red de la Comunidad de Regantes de Montesusín.
\item Tasa de decaimiento del cloro: 24 día$^{-1}$.
\end{itemize}

\FIGIV{24-2021-chlorine-constant-0.eps}{24-2021-chlorine-constant-2.eps}
{24-2021-chlorine-constant-6665.eps}{24-2021-chlorine-constant-172.eps}
{Concentración de cloro en distintos nodos con entrada constante. Año 2021}
{Fig242021ChlorineConstant}

\FIGIV{24-2022-chlorine-constant-0.eps}{24-2022-chlorine-constant-2.eps}
{24-2022-chlorine-constant-68.eps}{24-2022-chlorine-constant-172.eps}
{Concentración de cloro en distintos nodos con entrada constante. Año 2022}
{Fig242022ChlorineConstant}

\FIGV{24-2021-chlorine-day-0.eps}{24-2021-chlorine-day-2.eps}
{24-2021-chlorine-day-65.eps}{24-2021-chlorine-day-6665.eps}
{24-2021-chlorine-day-172.eps}
{Concentración de cloro en distintos nodos con entrada con una ventana horaria
diaria. Año 2021}{Fig242021ChlorineDay}

\FIGV{24-2022-chlorine-day-0.eps}{24-2022-chlorine-day-2.eps}
{24-2022-chlorine-day-6634.eps}{24-2022-chlorine-day-68.eps}
{24-2022-chlorine-day-172.eps}
{Concentración de cloro en distintos nodos con entrada con una ventana horaria
diaria. Año 2022}{Fig242022ChlorineDay}

\FIGV{24-2021-chlorine-week-0.eps}{24-2021-chlorine-week-2.eps}
{24-2021-chlorine-week-68.eps}{24-2021-chlorine-week-123.eps}
{24-2021-chlorine-week-172.eps}
{Concentración de cloro en distintos nodos con entrada con una ventana horaria
semanal. Año 2021}{Fig242021ChlorineWeek}

\FIGV{24-2022-chlorine-week-0.eps}{24-2022-chlorine-week-2.eps}
{24-2022-chlorine-week-6634.eps}{24-2022-chlorine-week-139.eps}
{24-2022-chlorine-week-172.eps}
{Concentración de cloro en distintos nodos con entrada con una ventana horaria
semanal. Año 2022}{Fig242022ChlorineWeek}

\clearpage
\subsection{Efecto de los venenos}

\FIGIIC{2022-mussel-constant.pdf}{Sin tratamiento}
{24-2021-chlorine-day.pdf}
{Tratamiento de 8 a 20}{$\approx$ -70\% Cl, -97,1\% M}

\FIGIIC{2022-mussel-constant.pdf}{Sin tratamiento}
{24-2021-chlorine-dayb.pdf}
{Tratamiento de 20 a 8}{$\approx$ -30\% Cl, -99,3\% M}

\FIGIIC{2022-mussel-constant.pdf}{Sin tratamiento}
{24-2021-chlorine-hour.pdf}
{Tratamiento hora sí hora no}{$\approx$ -50\% Cl, -98,9\% M}

\clearpage
\section{Modelo de crecimiento de adultos}

Hipótesis:
\begin{itemize}
\item Las larvas tardan un tiempo en convertirse en adultos una vez pegadas
(¿1 semana?)
\item El crecimiento viene a ser $\approx$ 1 cm / año.
\item La tasa de crecimiento está modulada por la temperatura.
\item El crecimiento se paraliza si la velocidad de flujo es muy alta.
\end{itemize}

\section{Conclusions}

\section*{Aknowledgments}

\bibliographystyle{elsarticle-harv}
\bibliography{bib}

\end{document}
