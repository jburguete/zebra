\documentclass[review,authoryear]{elsarticle}

\usepackage[utf8]{inputenc}
\usepackage[english]{babel}
\usepackage{lineno}
\usepackage{url}
\usepackage{multirow}
\usepackage[gen]{eurosym}
\usepackage{orcidlink}

\begin{document}

\begin{frontmatter}

\title{A coupled model of zebra mussels and chlorine in collective pressurized
	irrigation networks}

\author[1]{J. Burguete\,\orcidlink{0000-0003-4367-2598}}
\ead{jburguete@eead.csic.es}

\author[1]{B. Latorre\,\orcidlink{0000-0002-6720-3326}}
\ead{borja.latorre@csic.es}

\author[1]{P. Paniagua\,\orcidlink{0000-0003-0078-678X}}
\ead{pilucap@eead.csic.es}

\author[1]{E. T. Medina\,\orcidlink{0000-0002-7726-881X}}
\ead{emedina@eead.csic.es}

\author[1]{J. Fernández-Pato\,\orcidlink{0000-0002-3635-6223}}
\ead{jfernandez@eead.csic.es}

\author[1]{E. Playán\,\orcidlink{0000-0002-4808-7972}}
\ead{enrique.playan@csic.es}

\author[1]{N. Zapata\corref{cor1}\,\orcidlink{0000-0003-4632-7562}}
\ead{v.zapata@csic.es}
\cortext[cor1]{Corresponding author}

\affiliation[1]{organization={Soil and Water, EEAD, CSIC},
	addressline={Av. Montañana 1005},
	postcode={50059},
	city={Zaragoza},
	country={Spain}}

\date{\today}

\begin{abstract}
Zebra mussel infestation has become a serious problem affecting pressurized
collective irrigation networks. Mussel larvae settle in pipeline walls and
create relevant obstructions to flow as they grow and develop a shell. Chemical
treatments are commonly used to control the infestation, injecting oxidants
(frequently chlorine) in the irrigation stream. A model is proposed for
simulating the transportation, settlement and death of zebra mussels in
pressurized irrigation networks. This model is coupled to a solute transport,
diffusion, and decay model of oxidant chemicals used to control this invasive
species. During the analyzed irrigation campaigns, the entry of zebra mussel
larvae into a collective pressurized network was monitored at various intervals.
These data, showing high variability, were used as input to the model using
different scenarios averaging and modifying the time distribution of larvae
concentration. Model simulations consistently predicted similar mussel
settlement patterns across all scenarios, suggesting that network morphology and
total larval abundance primarily influence settlement distribution. We compared
the effectiveness of continuous and intermittent oxidant applications for mussel
control. Continuous treatments were the most effective (up to 99\%), but
required up to 3.5~kg$\cdot$ha$^{-1}$ of chlorine. Reasonable control could also
be attained with intermittent treatments, particularly with short injections (1
to 3 hours) just before the peak irrigation service discharge. Such treatments
led to up to 93\% of chlorine savings and reached mussel mortality rates close
to those of the continuous treatment. The model was also used to estimate the
larvae and chlorine export to on-farm irrigation systems through hydrants and to
evaluate strategies for mitigating the risks of on-farm infestation and
environmental impact. The protection of on-farm irrigation systems required
additional chlorine input. The proposed model can be parametrized to simulate
similar invasive species in different types of pressurized networks, using
different chemicals for treatment.
\end{abstract}

\begin{highlights}
\item
We introduced a model for zebra mussel dispersion, settlement and death by oxidants
\item
The distribution of settled mussels depends more on hydraulics than on larval input
\item
Short treatments are more effective when applied before the peak daily discharge
\item
On-farm protection needs more chlorine concentration than at the collective network
\end{highlights}

\begin{keyword}
zebra mussel, veligers, chlorine, solute transport, pressurized networks,
irrigation
\end{keyword}

\end{frontmatter}

\include{macro}

\linenumbers

\section*{Nomenclature}

{\scriptsize
\begin{tabular}{lcl}
$\Delta$ & = & temporal differences (s),\\
$\delta$ & = & spatial differences (m),\\
$\theta$ & = & implicit parameter (-),\\
$\sigma$ & = & surface density of mussels on the pipe wall
	(mussels$\cdot$m$^{-2}$),\\
$\tau$ & = & coefficient of mussel mortality (s$^{-1}$),\\
$\mathbf{\Psi}^\pm$ & = & upwind flux limiter matrices (-),\\
$\psi$ & = & flux limiter function (-),\\
$A$ & = & cross-sectional area (m$^2$),\\
$C$ & = & mussel larvae sources (larvae$\cdot$m$^2\cdot$s$^{-1}$),\\
$c$ & = & mussel larvae concentration (larvae$\cdot$m$^{-3}$),\\
$C_0$ & = & histogram of chlorine concentration at inlet (kg$\cdot$m$^{-3}$),\\
$c_0$ & = & initial mussel larvae concentration (larvae$\cdot$m$^{-3}$),\\
$d$ & = & death probability (-),\\
DOY & = & Day Of Year (days),\\
$\vec{F}$ & = & advection flux vector,\\
$f$ & = & friction factor of Darcy-Weissbach (-),\\
$k$ & = & constant of larval concentration (larvae$\cdot$m$^{-3}$),\\
$K_b$ & = & base solute decay coefficient (s$^{-1}$),\\
$K_w$ & = & wall solute decay coefficient, (m$\cdot$s$^{-1}$)\\
$K_x$ & = & longitudinal dispersion coefficient (m$^2\cdot$s$^{-1}$),\\
$M_e$ & = & total mass of free chlorine exported by the network (kg),\\
$N_e$ & = & number of mussel larvae exported by the network (larvae),\\
$N_i$ & = & number of settled mussels at the $i$-cell (mussels),\\
$N_l$ & = & total number of mussel larvae entering the network (larvae),\\
$N_t$ & = & total number of settled mussels at the network (mussels),\\
$P$ & = & perimeter (m),\\
$p$ & = & settlement probability (-),\\
$Q$ & = & discharge (m$^3\cdot$s$^{-1}$),\\
$R$ & = & pseudo-random number (-),\\
$r$ & = & correlation coefficient (-),\\
$S$ & = & solute sources (kg$\cdot$m$^{-1}\cdot$s$^{-1}$),\\
$s$ & = & solute concentration (kg$\cdot$m$^{-3}$),\\
$S_0$ & = & histogram of larval concentration at inlet (larvae$\cdot$m$^{-3}$),
	\\
$s_m$ & = & lethal solute concentration (kg$\cdot$m$^{-3}$),\\
$t$ & = & time (s),\\
TVD & = & Total Variation Dimishing,\\
$\vec{u}$ & = & vector of variables,\\
$V$ & = & larval settlement rate (m$\cdot$s$^{-1}$),\\
$v$ & = & flow velocity (m$\cdot$s$^{-1}$),\\
$x$ & = & longitudinal coordinate (m).\\
\end{tabular}
}

\section{Introduction}

Zebra mussels (\textit{Dreissena polymorpha}) are bivalve native to the Aral,
Black, and Caspian Sea basins of Europe. In its expansion to many areas of the
world, this has become one of the most damaging invasive species
\citep{USACE13}. Human activity in and around water bodies, the regulation of
rivers and the construction of conveyance structures have been very important
for the dispersion of the species and the colonization of new areas
\citep{Araujo06}. Zebra mussel spread was already documented in the 19th Century
\citep{Aldridge04}, but has become very intense in the 21st century.

The biology of zebra mussels was summarized by \citet{Cohen09}. Four larval
stages have been defined: embryonic, trochophore, veliger and pediveliger stage.
The veliger phase can last for one or more weeks, while the other three phases
may last for hours. The warmer and more productive the water, the shorter the
larval period. These stages are planktonic, meaning that larvae drift in water.
Larval capacity to swim in water is very limited, and can be considered
negligible. Pediveligers settle by attaching themselves to a substrate with
byssal threads, becoming juvenile mussels. Within a few months, mussels evolve
to sexually mature adults. Females can lay between forty thousand and a million
eggs annually \citep{CHE07}. Juveniles can drift in water currents, reaching
long distances. In the conditions of northern Spain, juveniles appear at the
beginning of summer, reaching high concentrations in water. A second peak has
been observed at the end of summer \citep{Araujo06, URA14}. Adults can reach up
to 1.5~cm in their first year of life, and a maximum size of 3~cm.

By attaching themselves to water structures, zebra mussels create many problems
to water managers and users. The species can obstruct water conveyance in
canals, gates, pipelines, valves and filters, challenging the continuity of
operation. Adults form colonies on substrates, sometimes piling up in layers. By
the end of the 20th century, in the United States, \citet{Khalanski97} estimated
the cost of this species’ invasion of water systems at
5~G\textdollar$\cdot$yr$^{-1}$. Costs have since increased in this and other
areas of the world.

Pressurized pipes are a common target for this invasive species.
\citet{MoralesHernandez18} documented the infestation of a 135~k~ha irrigation
project in the Ebro Valley of northeastern Spain. Over just five years, zebra
mussels spread to two-thirds of the project, with half of the area requiring
regular chemical treatment. Zebra mussels have rapidly colonized collective
pressurized irrigation networks because water is not routinely treated (unlike
in urban water systems) and because infestations remain undetected until
advanced stages. \citet{MoralesHernandez18, MoralesHernandez22} presented the
Normalized Pressure Method, oriented to the early detection and targeted
chemical treatment of zebra mussel infestation. The comparison of observed head
losses with those of the non-infested network has proven effective to identify
hot spots where mussels obstruct water passage. The method also detects the
presence of dead shells freely moving with water inside the pipes. Dead shells
create very serious problems in irrigation networks, collapsing filters dozens
of times a day. In severely infested areas, farmers have resorted to automate
filters for self-cleaning.

The control of zebra mussel infestation in collective pressurized networks can
be based on several strategies. The modification of the environment can be made
at low cost by desiccation of reservoirs in winter, removing adults if possible
and exposing them to dryness and low temperatures. Chemical treatments with
oxidants such as chlorine or hydrogen peroxide have proven very effective to
prevent the fixation of larvae (low concentration) or kill adults (high
concentration). Chlorine water disinfection has long been known to produce
byproducts toxic for the humans and the environment \citep{Hanigan17}.
Chlorination is widely used to reduce the risk of human diseases when applying
reclaimed irrigation water. Chlorination levels have also been assesed for
impact on plant growth and for accumulation in the soil. The level of
chlorination affecting plant growth is usually high (about
$20\cdot10^{-3}$~kg$\cdot$m$^{-3}$) and depends on plant sensitivity and
development stage. \citet{LiLi09} refer that water chlorination inhibits
nitrogen uptake in tomato plants without effect on crop yield.

There is no definitive control treatment if there is a continuous source of
larvae. As a consequence, subsequent treatments are required in many collective
irrigation networks, in the form of continuous low-concentration injections or
point injections with a high concentration of oxidants. Point injections kill
adults and dead shells are released in the water stream days or weeks after the
treatment.

Researchers have developed models to simulate zebra mussel population dynamics
and inform risk assessments. \citet{Wu10} adapted the Comprehensive Aquatic
Systems Model (CASM) to predict zebra mussel population growth, spatial
distribution, and ecological impacts in natural environments. The model paid
particular attention to the physical-chemical habitat requirements and to the
impacts on other species in the ecosystem. \citet{Huang17} presented a hybrid
model for the invasion dynamics of zebra mussel in rivers. The model was based
on the specific characteristics of this species, particularly on its negligible
capacity to move against the river current. The model addressed sources and
sinks of individuals, the reproduction process and dispersion in the river. This
model was expanded by \citet{JinZhao21}, who proposed a mathematical model of
zebra mussel spatial dynamics in river environments. The model was developed for
bounded and unbounded habitats, and considered elements of the mussel life
cycle.

In this research, a model is presented of the interaction between zebra mussels,
pressurized networks and the oxidants used for their chemical treatment.
Specific objectives include:
\begin{enumerate}
	\item Develop a numerical model simulating the transport and dispersion
		of zebra mussel larvae and the chemicals used for their
		treatment within the pipelines of collective pressurized
		irrigation networks.
	\item Evaluate the model's ability to analyze zebra mussel population
		dynamics under chemical treatment.
\end{enumerate}

\section{Materials and methods}

\subsection{Theorical models}

The proposed model has implemented some of the concepts developed for zebra
mussel transport and life cycle in rivers \citep{Wu10,Huang17,JinZhao21},
focusing on the process of larvae dispersion and settlement in closed conduits.

\subsubsection{Model of transport and dispersion of solutes in a pipe}

The equation governing the one-dimensional cross-sectional averaged transport
and dispersion of solutes in a pipe is the advection-diffusion Fick's law with
sources due to external injections or reaction decays:
\EQ{\PARTIAL{}{t}(A\,s)+\PARTIAL{}{x}(Q\,s)
	=S+\PARTIAL{}{x}\PA{K_x\,A\,\PARTIAL{s}{x}}}{EqTransportSolute}
where $t$ is the time, $A$ the cross-sectional area of the pipe, $s$ the solute
concentration, $x$ the longitudinal coordinate, $Q$ the discharge, $S$ the
solute source and $K_x$ the longitudinal dispersion coefficient.

In this work,  the $K_x$ coefficient was formulated as proposed by
\citet{JaviTrans} according to the \citet{Rutherford94} work: 
\EQ{K_x=10\,P\,|v|\,\sqrt{\frac{f}{2}}}{EqKx}  
with $P$ the perimeter of the pipe, $v$ the water velocity and $f$ the friction
factor of Darcy-Weissbach \citep{Darcy58}.

\subsubsection{Model of solute decay}

Two effects need to be considered to describe the solute decay in a pipe respect
to time: the reaction of the solute with the matter disolved in the water and
the reaction with the pipe walls. The following equation is used for this
purpose \citep{DiGianoZhang05}:
\EQ{\PARTIAL{s}{t}=-\PA{K_b-K_w\,\frac{P}{A}}\,s}{EqSoluteDecay}
with $K_b$ and $K_w$ the base and wall decay constants, respectively. 

\subsubsection{Model of transport and dispersion of mussel larvae in a pipe}

Given the negligible swimming capacity of zebra mussel larvae compared to water
velocity in pressurized networks, they are assumed to behave as neutrally
buoyant particles within the flow. Consequently,
Equation~(\ref{EqTransportSolute}), adapted for mussel larvae, can be used to
model their transport and dispersion through pipes:
\EQ{\PARTIAL{}{t}(A\,c)+\PARTIAL{}{x}(Q\,c)
	=C+\PARTIAL{}{x}\PA{K_x\,A\,\PARTIAL{c}{x}}}{EqTransportLarvae}
with $c$ the larvae concentration and $C$ the sources of larvae.

\subsubsection{Model of mussel larvae settlement to a pipe}

\citet{Cohen05} reported that \citet{Boelman97} found that zebra mussel larvae
are unable to attach themselves to rivers with flow velocities exceeding
approximately 1.5-2.0~m$\cdot$s$^{-1}$. He also reported that
\citet{SmirnovaVinogradov90} indicated that larvae reduce their feeding rates at
velocities greater than 1.0-1.5~m$\cdot$s$^{-1}$ as the flow deforms the
mussel's siphon. Based on these studies, we have imposed a maximum velocity of
1.5~m$\cdot$s$^{-1}$, beyond which larvae will not attach to the pipe walls. On
the other hand, a dimensional study suggests that larval settlement to the pipe
should follow a law of the form:
\EQ{\PARTIAL{c}{t}=-\frac{V\,P\,c}{A},\quad
	\PARTIAL{\sigma}{t}=V\,c}{EqMusselSettlement}
with $V$ the settlement rate, and $\sigma$ the surface density of mussels on
the pipe wall.

\citet{Alix16} and \citet{AlonzoMoya21} investigated zebra mussel larval
settlement in a lake and multiple water treatment plants. Their studies revealed
a highly variable process, with larval settlement rates fluctuating widely and
showing little correlation with temperature or water column larval density.
Consequently, estimating larval settlement rates is inherently uncertain. Our
settlement models are therefore expected to provide qualitative trends rather
than precise predictions of mussel surface densities. From the second equation
in (\ref{EqMusselSettlement}), we can estimate the settlement rate as the ratio
between surface density and the product of larval concentration and the duration
of the settlement season. Based on the data from \citet{Alix16} and
\citet{AlonzoMoya21}, with typical values of 
$c\approx 10^3$~larvae$\cdot$m$^{-3}$,
$\sigma\approx 10^2$~mussels$\cdot$m$^{-2}$ and a settlement time of
$\Delta t\approx4$~months~$\approx10^7$~s, we can estimate a value of
$V\approx\frac{\sigma}{\rho\,\Delta t}\approx10^{-8}$~m$\cdot$s$^{-1}$. We
selected representative values for $c$ and $\sigma$ because it can be misleading
to use arithmetic means for values that may vary by three orders of magnitude.

\subsubsection{Model of mussel death by chlorine\label{SecMusselDeath}}

The mortality of zebra mussels depends upon chlorine concentration, contact time
with the oxidant and water temperature \citep{vanBenschoten95}.
\citet{SprecherGetsinger00} reported that the relationship between concentration
and exposure time was typically inversely linear, achieving 100\% larval
mortality with a chlorine concentration of 0.5~kg$\cdot$m$^{-3}$ in 2 hours.
However, for adults, a concentration of 2.0~kg$\cdot$m$^{-3}$ was required.
Based on these data, and considering that the water temperature in buried
pressurized irrigation networks varies relatively little during the irrigation
season (see Figure~\ref{FigMeasured}), we adopted the following model to
describe larval density evolution in the presence of chlorine:
\EQ{\PARTIAL{c}{t}=-\frac{\tau\,s\,c}{s_m},
\quad\PARTIAL{\sigma}{t}=-\frac{\tau\,s\,\sigma}{s_m}}{EqMusselDecay}
with $s$ the current free chlorine concentration, $s_m$ the lethal free
chlorine concentration ($0.5\cdot10^{-3}$~kg$\cdot$m$^{-3}$ for larvae and
$2\cdot10^{-3}$~kg$\cdot$m$^{-3}$ for adults, as previously reported).
Once settled, the transition from larval to adult stage was assumed to occur
within one week. Equation (6) yields an analytical solution for $c$:
\EQ{c=c_0\,\exp\PA{-\frac{\tau\,s\,\Delta t}{s_m}}}{EqMusselDecayAnalytic}
with $c_0$ the initial larvae concentration. To determine the value for $\tau$,
we approximate in this work that in the experiments described in
\cite{SprecherGetsinger00}, 99\% larval mortality rate was achieved. In this
case:
\EQ{\exp\PA{-7200\,\tau}=0.01\Rightarrow\;\tau=6.4\cdot10^{-4}}
{EqMusselDecayTau}

\subsection{The numerical model}

The equations above were applied to the geometry of collective pressurized
pipeline networks, composed of interconnected buried pipelines, one or more
sources of water, larvae and oxidants, and hydrants of known demand discharge.
The oxidant chemical injected in water to control mussel population was assumed
to be chlorine. This assumption was required to input specific mortality
concentrations in the model.

To accurately solve the solutes and mussel larvae concentration equations,
each pipe is discretized in a set of cells with their centers located at a
position $x_i$, where $x_i$ is the longitudinal coordinate along the pipe. A
cell size of $\delta x_i=10$~m was used. A split-step approach was employed to
address the various processes involved. We defined $\vec{u}_i^n$ as the vector
of variable values in the $i$-th cell and in the $n$-th time step. In each time
step, the variable increments $\Delta\vec{u}_i^n=\vec{u}_i^{n+1}-\vec{u}_i^n$
are divided into advection-diffusion~(A), solute decay~(D),
larvae settlement~(S), mussel mortality~(M), junctions~(J), and inputs~(I)
substeps:
\EQ{\Delta\vec{u}_i^n=\Delta\vec{u}_i^A+\Delta\vec{u}_i^D+\Delta\vec{u}_i^S
	+\Delta\vec{u}_i^M+\Delta\vec{u}_i^J+\Delta\vec{u}_i^I}{EqSplit}
where $\vec{u}_i^A=\vec{u}_i^n+\Delta\vec{u}_i^A$,
$\vec{u}_i^D=\vec{u}_i^A+\Delta\vec{u}_i^D$,
$\vec{u}_i^S=\vec{u}_i^D+\Delta\vec{u}_i^S$,
$\vec{u}_i^M=\vec{u}_i^S+\Delta\vec{u}_i^M$,
$\vec{u}_i^J=\vec{u}_i^M+\Delta\vec{u}_i^J$, and
$\vec{u}_i^{n+1}=\vec{u}_i^J+\Delta\vec{u}_i^I$ represent the variable values
after the advection-diffusion, solute decay, larvae settlement, mussel
mortality, junctions, and inlet boundary conditions substeps, respectively.

\subsubsection{Hydraulic model}

The EPANET software \citep{Rossman94} was used to determine the pressure at the network nodes and the discharge at the network pipes to satisfy hydrant
discharge demands for given head at the network inlet points. The EPANET output
was used as input of ad hoc advection-diffusion models for the oxidant chemical
and larvae.

\subsubsection{Model of advection-diffusion equation}

The Roe-Sweby \citep{Sweby84} second order in space and time TVD (total
variation diminishing) scheme is used to solve the advection term combined with
a centred implicit scheme of second order in space to solve the diffusion term
of the advection-diffusion equations (\ref{EqTransportSolute}) and
(\ref{EqTransportLarvae}):
\[\vec{u}=\MATRIX{c}{c\\s},\quad
	\delta\vec{u}_{i+\frac12}=\vec{u}_{i+1}-\vec{u}_i,\quad
	\delta\vec{F}_{i+\frac12}^n=\PA{1-v\,\frac{\Delta t}{\delta x}}\,v\,
	\delta\vec{u}_{i+\frac12}^n,\]
\[\delta\vec{F}^\pm=\frac{v\pm|v|}{2}\,\delta\vec{u}_{i+\frac12},\quad
	\psi(r)=\max\CO{0,\;\min\PA{2\,r,\;\frac{1+r}{2},\;2}},\]
\[
	\mathbf{\Psi}_{i+\frac12}^\pm=\MATRIX{cc}{
	\psi\PA{\frac{\PA{\delta\vec{F}_{i+\frac12\pm1}^\pm}_1}
	{\PA{\delta\vec{F}_{i+\frac12}^\pm}_1}} & 0\\
	0 & \psi\PA{\frac{\PA{\delta\vec{F}_{i+\frac12\pm1}^\pm}_2}
	{\PA{\delta\vec{F}_{i+\frac12}^\pm}_2}}},\]
\[-\theta\,K_x\,\frac{\Delta t}{\delta x}\,\Delta\vec{u}_{i-1}^A
	+\PA{\delta x+2\,\theta\,K_x\,\frac{\Delta t}{\delta x}}
	\,\Delta\vec{u}_i^A
	-\theta\,K_x\,\frac{\Delta t}{\delta x}\,\Delta\vec{u}_{i+1}^A\]
\[=(1-\theta)\,K_x\,\frac{\Delta t}{\delta x}\,\PA{\delta\vec{u}_{i+\frac12}^n
	-\delta\vec{u}_{i-\frac12}^n}\]
\[+\Delta t\,\left\{\frac12\,\PA{\mathbf{\Psi}^+\,\delta\vec{F}^+}_{i-\frac32}^n
	-\CO{\PA{1+\frac12\,\mathbf{\Psi}^+}
	\,\delta\vec{F}^+}_{i-\frac12}^n\right.\]
\EQ{\left.-\CO{\PA{1+\frac12\,\mathbf{\Psi}^-}\,\delta\vec{F}^-}_{i+\frac12}^n
	+\frac12\,\PA{\mathbf{\Psi}^-\,\delta\vec{F}^-}_{i+\frac32}^n\right\}}
	{EqAdvectionDiffusion}
with $\vec{F}$ the advection flux, $\PA{\delta\vec{F}^\pm}_k$ the $k$-th
component of the vector $\delta\vec{F}^\pm$, $\theta\in[0,1]$ the implicit
parameter, $\psi$ the monotonized central flux limiter function
\citep{vanLeer77}, and $\delta$ for spatial differences. This scheme produces a
tridiagonal system of equations in $\Delta\vec{u}_i^A$ that is solved by
Gaussian elimination.

For $\theta=\frac12$ the diffusion scheme is second order in time. According to
\citet{JaviTesis}, the scheme maintains TVD properties if the following
conditions are met:
\EQ{\Delta t\leq\min\PA{\frac{\delta x}{|v|}},\quad
	1\geq\theta\geq1-\frac{2\,\delta x^2}{K_x\,\Delta t}}{EqCFL}
being the first the well known Courant-Friedrichs-Lewy condition \citep{CFL} and
the second a condition on the implicit parameter. To balance accuracy and
computational efficiency while preserving TVD characteristics, the following
parameter values were employed:
\EQ{\Delta t=0.9\,\min\PA{\frac{\delta x}{|v|}},\quad
	\theta=\max\PA{\frac12,\;1-\frac{\delta x^2}{2\,K_x\,\Delta t}}}
{EqTheta}

\subsubsection{Model of solute decay}

The solute decay equation (\ref{EqSoluteDecay}) can be solved analytically
within a time step, resulting in:
\EQ{s_i^J=s_i^A\,\exp\CO{-\PA{K_b-K_w\,\frac{P}{A}}\,\Delta t}}{EqStepDecay}

\subsubsection{Model of mussel larvae settlement to a pipe}

Since the settlement process of larvae is fundamentally stochastic, the model
uses a random pattern to determine settlement. According to
(\ref{EqMusselSettlement}), the probability of a larva settlement to a cell in
the pipe during a time step is in first order:
\EQ{p_i^S=-A\,\delta x\,\Delta c_i^S=V\,P\,c_i^J\,\Delta t\,\delta x}
{EqMusselSettlementProbability}
We generate a pseudo-random number $R\in[0,1)$ and if $R<p_i^S$, we add an
element to the list of mussels associated with the cell. Each element stores the
time at which settlement occurrs to determine the transition from larva to adult
stages. To conserve the number of larvae, each time a new element is added, the
larval concentration must be adjusted accordingly:
\EQ{c_i^S=\max\PA{0,\;c_i^J-\frac1{A\,\delta x}}}
{EqMusselSettlementConcentration}
Additionally, if $N_i^S$ is the number of mussels in the cell list, the surface
density of mussels will be:
\EQ{\sigma_i^S=\frac{N_i^S}{P\,\delta x}}{EqMusselSettlementDensity}
This stochastic method only makes sense if $p_i^S<1$. This leads to the
following restriction on the time step:
\EQ{\Delta t<\frac{1}{V\,P\,c_i^J\,\delta x}}{EqMusselSettlementTimeStep}

\subsubsection{Model of mussel death by chlorine}

Equation~(\ref{EqMusselDecayAnalytic}) offers an analytical solution for larval
concentration evolution over a time step. Applying this to our model yields:
\EQ{c_i^M=c_i^S\,\exp\PA{-\frac{\tau\,s\,\Delta t}{s_m}}}{EqStepDeath}

The numerical process of mussel mortality of settled mussels is also considered
stochastic. According to (\ref{EqMusselDecay}), the probability of a mussel
dying ($d_i^n$) in the presence of free chlorine during a time step is in first
order:
\EQ{d_i^M=\frac{\Delta\sigma_i^M}{\sigma_i^S}=\frac{\tau\,s_i^S\,\Delta t}{s_m}}
{EqMusselDecayProbability}
where the value of $s_m$ depends on whether the mussel is in the larval or
adult stage, as detailed in subsection~\ref{SecMusselDeath}. For each mussel in
the list associated with each cell in the network's pipes, we generate a
pseudo-random number $R\in[0,1)$. If $R<d_i^M$, the mussel is removed from the
list and $\sigma_i^M$ readjusted as in (\ref{EqMusselSettlementDensity}). As in
(\ref{EqMusselSettlementTimeStep}) the probability $d_i^M$ has to be less than
one, the following time step restriction applies:
\EQ{\Delta t<\frac{s_m}{\tau\,s_i^S}}{EqMusselDeathTimeStep}

\subsubsection{Model of junctions}

To solve the solute or larval concentration at network junctions, equality of
concentrations and mass conservation at all nodes involved in the junction is
imposed:
\EQ{\vec{u}^J=\frac{\sum_jA_j\,\vec{u}_j^M\,\delta x_j}{\sum_jA_j\,\delta x_j}}
{EqJunction}
with $\vec{u}^J$ the vector of variables at all the nodes and $j$ for each node
involved in the junction.

\subsubsection{Model of inlets}

The numerical scheme requires setting boundary conditions at the inlet for
chlorine and larvae concentrations. Specifically, for the inlet at the 0-th
point, we impose:
\EQ{s_0^{n+1}=S_0\PA{t^{n+1}},\quad c_0^{n+1}=C_0\PA{t^{n+1}}}{EqInlet}
with $S_0$ and $C_0$ being the time-varying chlorine and larvae concentrations,
respectively, at the entry point.

\subsubsection{Initial conditions}

Finally, to complete the numerical scheme, initial conditions are required for
chlorine concentration, larvae concentration, and the number of settled mussels
in each cell of the network pipes. In this study, it was assumed that there was
no chlorine, larvae, or mussels present at the start of each campaign.

\subsection{A case study: the CS-WUA \#1 irrigation network}

The Collarada Segunda Water Users Association (CS-WUA), located in Montesusín
(Huesca, Spain), was selected for this study. This WUA makes part of the Riegos
del Alto Aragón project, which has been affected by zebra mussel since 2013
\citep{MoralesHernandez18}. The transition from surface irrigation to sprinkler
/ drip irrigation, and consequently from canals and ditches to pressurized
pipelines has made this project vulnerable to zebra mussel. Although the system
reservoirs have long been infested with zebra mussel, the use or pipelines for
irrigation has strongly increased the impact of zebra mussel in the area.

The CS-WUA covers an irrigated area of 3,126~ha. The pressurized area is
irrigated with two independent networks. Network \#1 was used for this study.
The network irrigates about 1,700~ha. Water is pumped from the main canal to an
elevated reservoir (41~m above the pump station). Water flows by gravity from
this reservoir to the network hydrants. The network has 123 hydrants and 208
pipelines. The diameter of the pipes ranges from 131~mm to 1,400~mm, with a
total pipe length of 46.6~km. The irrigation network operates in arranged demand
mode \citep{Clemmens87} to ensure timely water delivery to farmers while
minimizing electricity consumption for reservoir filling.

\section{Results and discussion}

In the following sections, experimental data is presented documenting the
boundary conditions of model simulation. Additionally, the model is applied to
assess the relationship between the larvae input and zebra mussel infestation
risk, to assess the effectiveness of chemical treatments in the control of
this invasive species and to evaluate the export of larvae and chemicals through
the network hydrants.

\subsection{CS-WUA \#1 network: boundary conditions}

Figure~\ref{FigMeasured} depicts water temperature measurements taken at a
network manhole in 2022 and 2023 and the daily evolution of concentration of
zebra mussel larvae at the inlet of the network for three different campaigns.
It is clear that temperatures in excess of 10$^\circ$C are reached early in the
season, facilitating zebra mussel spawning. Moreover, daily temperature
fluctuations are minimal. Throughout the irrigation season, spanning from early
May to early October, water temperatures remained relatively stable, typically
ranging between 20 and 25$^\circ$C. For each larval concentration measurement,
100 liters of water were passed through a filter, and then the filter was
analyzed to count the number of larvae. Larval counts were produced at the Zebra
Mussel Laboratory of the OX-CTA company (\url{https://grupoox.com}). In each
campaign, measurements were made at different times of the year. A strong random
character is observed in this variable, with sudden, strong variations of up to
three orders of magnitude. It was not possible to detect relationships with
temperature or date of measurement which could have explained such variations.
These results are consistent with other studies showing strong variability in
larval presence, with low correlations between larvae density, adult population
or temperature \citep{AlonzoMoya21}.

\FIGIII{Figure_1a.pdf}{Figure_1b.pdf}{Figure_1c.pdf}
{Water temperature measured at a manhole in the years 2022 and 2023 (top) and
larval density measured at the inlet at different times over three different
years (bottom) of the CS-WUA \#1 network versus Day of the Year (DOY). The
strong time variability of larval density did not respond to water temperature
or to daily hour of sampling}{FigMeasured}

In Figure~\ref{FigQD}, the time evolution of irrigation water demand (discharge
at the network inlet) and the hourly average of irrigation water demand are
depicted for the irrigation seasons 2021 and 2022 during the larval
treatment season (from mid-May to mid-August). Demand increases from spring to
summer. A strong variation of discharge can be appreciated within each day. A
high demand is observed during nighttime hours, with the peak flow occuring
around 2:00~h in the morning. This is due to agronomic factors (reduced wind
drift and evaporation losses; increased uniformity) and to economic
considerations (lower electricity prices during off-peak hours). Inlet discharge
follows a very similar pattern in both years, with slightly higher discharge in
2021 than in 2022.

\FIGIII{Figure_2a.pdf}{Figure_2b.pdf}{Figure_2c.pdf}
{Time evolution of irrigation water demand (top) and inlet discharge averaged by
hours (bottom) in the CS-WUA \#1 network in the 2021 and 2022 campaigns during
the larval treatment season (form mid-May to mid-August). Demand shows relevant
fluctuations during the irrigation season and also within each irrigation day}
{FigQD}

\subsection{Effect of different larval input scenarios on risk of network
	infestation}

Next, we analyze the effect that the highly variable larval inputs could have on
the CS-WUA \#1 network infestation. To do this, we simulated four larval input
scenarios during two irrigation campaigns, 2021 and 2022, keeping the total
number of larvae in each campaign as measured:
\begin{enumerate}[I.]
\item larval input as measured,
\item constant larval input during the time windows of measurement (from
	mid-July to mid-August in 2021 and from mid-June to early-July in 2022),
\item larval input with random hourly variation of up to three orders of
	magnitude during the time windows of measurement, and
\item uniform larval input throughout the entire treatment season (from mid-May
	to mid-August).
\end{enumerate}
The total number of larvae $N_l$ entering the network was computed as:
\EQ{N_l=\sum_iQ_i\,c_i\,\Delta t_i}{EqLarvaeNumber}
where $Q_i$, $c_i$ y $\Delta t_i$ represent respectively the flow rates, larval
concentrations, and hourly time intervals throughout the entire campaign. In
scenario I, the concentration at each hour was temporally interpolated between
the measurements. In scenario III, to achieve variations of three orders of
magnitude while maintaining the number of larvae, concentrations were generated
using the formula:
\EQ{c_i=\frac{k}{0.001+R_i},\quad
	k=\frac{N_l}{\sum_i\frac{Q_i\,\Delta t_i}{0.001+R_i}}}{EqLarvaeRandom}
with $k$ being a constant and $R_i\in[0,1)$ pseudo-random numbers. For
the constant larval concentration scenarios, II and IV:
\EQ{c_i=\frac{N_l}{\sum_iQ_i\,\Delta t_i}}{EqLarvaeConstant}
The variables used in each scenario are detailed in Table~\ref{TabLarvaeInput}.
In the 2022 campaign, the number of larvae input was roughly double than that of
the 2021 campaign. Note the large number of estimated larvae entering the system
in each campaign, on the order of one billion.
\TABLE{cc|ccc}
{
	Year & Scenario & $N_l$ (larvae) & $c_i$ (larvae$\cdot$m$^{-3}$)
	& $k$ (larvae$\cdot$m$^{-3}$) \\
	\hline
        \multirow{4}{*}{2021} & I & $8.3\cdot10^8$ & - & - \\ 
        & II & $8.3\cdot10^8$ & 361.7 & - \\
        & III & $8.3\cdot10^8$ & - & 70.54 \\ 
        & IV & $8.3\cdot10^8$ & 139.9 & - \\
	\hline
        \multirow{4}{*}{2022} & I & $1.6\cdot10^9$ & - & - \\ 
        & II & $1.6\cdot10^9$ & 1,407 & - \\
        & III & $1.6\cdot10^9$ & - & 264.3 \\
        & IV & $1.6\cdot10^9$ & 295.5 & -
}{Variables used to generate the four scenarios of larval input in the 2021 and
2022 campaigns at the CS-WUA \#1 network}{TabLarvaeInput}

The measured and simulated larval inputs over time are presented in
Figure~\ref{FigMusselInput2022}, for the four scenarios in the 2021 and 2022
campaigns.

\FIGVIII{Figure_3a.pdf}{Figure_3b.pdf}{Figure_3c.pdf}{Figure_3d.pdf}
{Figure_3e.pdf}{Figure_3f.pdf}{Figure_3g.pdf}{Figure_3h.pdf}
{Measured and simulated larval density input at the network inlet over time
(years 2021 and 2022). Each subfigure corresponds to a different scenario: I,
larval input as measured; II, constant larval input during the time window of
measurement; III, larval input with random hourly variation of up to three
orders of magnitud; and IV, uniform larval input throughout the entire treatment
season\label{FigMusselInput2022}}

Maps of simulated infestation are shown for the four proposed scenarios of
larval inputs in the year 2022 (Figure~\ref{Fig2022MusselInput}) and in the year
2021 (Supplementary Figure~S.1). The thickness of the lines on the maps is
proportional to the surface density of mussels. Above each map, terms of
comparison are presented respect to the scenario IV results: the increase in
total settled mussels in the network ($N_t$) is presented as a percentage, along
with the correlation coefficients $r_1$, comparing the surface density of
settled mussels cell by cell in the network, and $r_2$, comparing the surface
density of settled mussels averaged for each pipe. Scenario IV exhibits the
highest infestation and a more uniform distribution. This is because it
introduces a relevant part of the larvae during weeks of low water demand. When
the network demand is high, flow velocities increase, giving the larvae less
time to settle to pipe walls before being discharged through the hydrants.
The effect is more pronounced in the 2021 campaign because the measurement
window, used in scenarios I-III, is coincident with the peak demand period.
Additionally, the likelihood increases that in some pipes velocity will exceed
the maximum limit that allows for larval settlement. The differences in settled
mussels do not exceed 17\% in 2022 and 45\% in 2021. The difference in larval
entry patterns between scenarios mainly affects infestation densities at distal
pipelines, while there are hardly any variations at the main lines. Correlation
coefficients are at least 0.62 for $r_1$ and 0.67 for $r_2$, indicating that the
infestation distribution is quite consistent across the different scenarios.
Given the difficulty in precisely introducing larval density due to the large
observed variations, and considering that the effect is not too significant
compared to other sources of uncertainty, we will use scenario IV with constant
larval density at the inlet in the following simulation cases.

\FIGIVB{Figure_4a.pdf}{$\Delta N_t=-16.8\%$, $r_1=0.62$, $r_2=0.67$}
{Figure_4b.pdf}{$\Delta N_t=-4.0\%$, $r_1=0.68$, $r_2=0.80$}
{Figure_4c.pdf}{$\Delta N_t=-9.1\%$, $r_1=0.74$, $r_2=0.79$}
{Figure_4d.pdf}{$\Delta N_t=0\%$, $r_1=1$, $r_2=1$}
{Maps of simulated infestation for scenarios I (top-left), II (top-right), III
(bottom-left), and IV (bottom-right) for the year 2022. For each subfigure,
indicators are presented for the increase in the number of total settled mussels
and correlation coefficients ($r_1$ for pipeline cells and $r_2$ for complete
pipelines) respect to scenario IV. The network pipeline thickness is
proportional to the surface density of settled
mussels\label{Fig2022MusselInput}}

To assess the impact of randomness in the stochastic models used for larvae
settlement and mussel mortalities, the simulation of scenario IV at the 2021
campaign was repeated using two different seeds for the pseudo-random number
generator. Next, we analyzed the influence of two parameters that have high
uncertainty: the number of larvae entering the system and the settlement rate.
To do this, we simulated scenario IV first by doubling the larvae concentration
at the inlet and second by doubling the settlement rate.
Figure~\ref{Fig2021MusselInput2} shows the resulting infestation maps, as well
as the parameters for variation in the number of settled mussels and the
correlation coefficients of surface density respect to the default seed, input
and settlement rate. Modifying the random seed, the variation in the number of
mussels is minimal, with the largest variation being 0.03\%. The correlation
coefficients in both cases are 0.88 for $r_1$ and 0.96 for $r_2$. Therefore, the
effect of randomness is slightly noticeable in $r_1$ and considerably lower in
$r_2$. Varying the input or the settlement rate, a linear effect is observed in
both cases, with the number of total settled mussels doubling. However, the
correlation coefficients are very high, indicating that these parameters barely
affect the distribution pattern of the infestation across the network.

\FIGIVB{Figure_5a.pdf}{$\Delta N_t=-0.02\%$, $r_1=0.88$, $r_2=0.96$}
{Figure_5b.pdf}{$\Delta N_t=-0.03\%$, $r_1=0.88$, $r_2=0.96$}
{Figure_5c.pdf}{$\Delta N_t=100\%$, $r_1=0.80$, $r_2=0.95$}
{Figure_5d.pdf}{$\Delta N_t=95\%$, $r_1=0.82$, $r_2=0.95$}
{Maps of simulated infestation of the CS-WUA \#1 network for scenario IV with
two different seeds for the pseudo-random numbers generator (top), doubling the
larvae concentration at the inlet (bottom-left) and doubling settlement rate
(bottom-right) for the year 2021. For each subfigure, indicators are presented
for the increase in the number of total settled mussels and correlation
coefficients ($r_1$ for pipeline cells and $r_2$ for complete pipelines) respect
to the default seed, input and settlement rate. The network pipeline thickness
is proportional to the surface density of settled
mussels\label{Fig2021MusselInput2}}

\subsection{Simulation of chlorine treatments}

In this section, the results of the interaction between mussels and injected
chemical in the CS-WUA \#1 network are presented. Chemical treatments with
oxidant agents (chlorine and/or hydrogen peroxide) have been used to control the
population of zebra mussels inside the sprinkler irrigation pipes. The specific
case of free chlorine application is analyzed in this section. In order to keep
the target free chlorine concentration in the irrigation water, WUAs should use
proportional injection pumps connected to a digital flow meter.

Supplementary Figure~S.2 presents the experimental free chlorine decay in time
when applied to the local water (obtained at the CS-WUA reservoir) and distilled
water. A base solute decay coefficient of $K_b=3.0\cdot10^{-4}$~s$^{-1}$ was
derived for the local water. Since the pipes of the CS-WUA \#1 network are
almost exclusively made of PVC, the following wall solute coefficient, extracted
from figures presented by \citet{AlJasser07}, has been used:
$K_w=1.4\cdot10^{-6}$~m$\cdot$s$^{-1}$.

Next, we analyze the propagation of chlorine through the network at three
different points in the network for the 2021 irrigation campaign. Point 0
represents the network inlet, Point 2 is located around the center of the
network in the main pipeline, and Point 119 is located at the distal end of a
line far from the inlet. In Figure~\ref{FigQCl}, the simulated evolution of
discharge and free chlorine concentration is presented at the three points, as
well as the location of these points in the network. A constant free chlorine
concentration of $1.0\cdot10^{-3}$~kg$\cdot$m$^{-3}$, double of the lethal
larvae dose, was injected at the inlet. It can be observed that even at point 2,
only with the highest flow rates a chlorine concentration above the larval
lethal dose was reached. At points far from the inlet, such as 119, chlorine
never reached the required concentration to damage the larvae. Therefore, an
interruption in the treatment or an incidental reduction in the concentration
will severely increase the infestation risk. This is due to the fact that free
chlorine does not reach sufficient concentration at points far from the inlet
unless injected at very high concentrations.

\FIGVII{Figure_6a.pdf}{Figure_6b.pdf}{Figure_6c.pdf}{Figure_6d.pdf}
{Figure_6e.pdf}{Figure_6f.pdf}{Figure_6g.pdf}
{Time evolution of discharge (left) and free chlorine concentration (right)
simulated at points 0, 2 and 119 of the CS-WUA \#1 network during the 2021
irrigation campaign with a continuous injection of free chlorine at the inlet at
a concentration of $1.0\cdot10^{-3}$~kg$\cdot$m$^{-3}$. The lethal free
chlorine concentration for larvae is also plotted as dashed lines for reference.
The location of the points in the network, represented with circles, is shown at
the bottom. Point 0 corresponds to the network inlet}{FigQCl}

Figure~\ref{Fig2021ChlorineConstant} shows the infestation maps of settled
zebra mussels simulated in the network for the 2021 irrigation campaign
(and Supplementary Figure~S.3 for the 2022 irrigation campaign), with constant
free chlorine injections at different concentrations:
$1.0\cdot10^{-3}$~kg$\cdot$m$^{-3}$ (double the lethal larval dose),
$0.5\cdot10^{-3}$~kg$\cdot$m$^{-3}$ (equal to the lethal larval dose), and two
sublethal larval doses of $0.25\cdot10^{-3}$~kg$\cdot$m$^{-3}$ and
$0.1\cdot10^{-3}$~kg$\cdot$m$^{-3}$. At the top of each subfigure, we present
the increase in the number of settled mussels compared to the scenario without
treatment, along with the amount of consumed free chlorine. With the highest
dose, infestation reductions of 99.9\% are achieved, although the cost of this
amount of free chlorine (up to 5,906 kg $\approx$ 3.5~kg$\cdot$ha$^{-1}$
$\approx$ 25~\euro{}$\cdot$ha$^{-1}$) may be unaffordable both economically and
environmentally. The lethal dose also achieves mitigations above 99\% saving
50\% of chlorine. With sublethal doses, the treatment's effectiveness
progressively degrades, but it is noteworthy that reductions of about 95\% in
infestation can be achieved with 90\% chlorine savings.

\FIGIVB{Figure_7a.pdf}{$\Delta N_t=-99.9\%$, Cl=5,906 kg}
{Figure_7b.pdf}{$\Delta N_t=-99.3\%$, Cl=2,953 kg}
{Figure_7c.pdf}{$\Delta N_t=-97.9\%$, Cl=1,476 kg}
{Figure_7d.pdf}{$\Delta N_t=-95.5\%$, Cl=591 kg}
{Maps of simulated infestation of mussels for scenario IV of the 2021 campaign,
using constant free chlorine injection at the inlet with concentrations of
$1.0\cdot10^{-3}$~kg$\cdot$m$^{-3}$ (top-left),
$0.5\cdot10^{-3}$~kg$\cdot$m$^{-3}$ (top-right),
$0.25\cdot10^{-3}$~kg$\cdot$m$^{-3}$ (bottom-left) and
$0.1\cdot10^{-3}$~kg$\cdot$m$^{-3}$ (bottom-right). At the top of each
subfigure, the increment of total settled mussels compared to the scenario
without treatment as well as the amount of free chlorine consumed are
shown\label{Fig2021ChlorineConstant}}

In an additional effort to reduce chlorine consumption, and considering the
observed pattern of strong hourly dependence in the CS-WUA \#1 network (refer to
Figure~\ref{FigQD}), the efficacy of different intermittent chlorine treatments
was assesed. These treatments involve injecting
$1.0\cdot10^{-3}$~kg$\cdot$m$^{-3}$ of free chlorine, double of the larval
lethal dose, at the inlet, but differ in the application duration and
starting times. Specifically, treatments lasting for 1, 2, 3, 6, and 12 hours,
each starting at every hour of the day, have been simulated for the 2021 and
2022 campaigns.

In Figure~\ref{FigMT}, the percentage of surviving settled mussels is presented
for both campaigns and for the five treatment durations as a function of the
starting time of the treatment. A strong dependence of treatment efficacy on the
treatment starting time is observed, with this dependence being much stronger
for short treatments. On the other hand, in the one-hour treatments, it is
observed that the highest mitigation occurs with treatments from 0:00 to 1:00~h
and from 23:00 to 24:00~h. Considering that the average peak flow occurs at
2:00~h, the highest effectiveness is achieved by injecting the product in the
hours immediately preceding the peak flow. This makes sense because high flow
rates induce high flow velocities in the pipes. As a consequence, chlorine
travels far and with high concentrations before decaying. The contrary can be
said about treatments performed at low input discharge. It is also observed that
long-duration treatments result in high settled mussel mitigation, although
they are also associated with high chlorine consumption.

\FIGII{Figure_8a.pdf}{Figure_8b.pdf}
{Simulated percentage of surviving settled mussels vs. treatment starting time
for different durations of chlorine treatment in the 2021 (left) and 2022
(right) irrigation seasons}{FigMT}

In Table~\ref{TabMClOptimal}, we present the optimal intermittent treatments for
achieving the highest settled mussel mitigation rates across different
treatment durations and starting times for the 2021 and 2022 irrigation seasons.
With the peak flow occurring at 2:00~h, all short-duration treatments
(ranging from 1 to 3 hours) achieved maximum reduction by concluding just
before reaching the peak. Daily interruption of treatment may be unsafe in
network areas where chlorine concentrations are insufficient to damage the
larvae. All simulated intermittent treatments show settled mussel survival rates
higher than 4.9\%. However, short-duration treatments still yield considerable
mitigation rates while offering substantial savings in chlorine consumption.
Notably, the optimal one-hour treatment emerges as a cost-effective,
environmentally friendly option, achieving mitigation rates above 90\% with a
remarkable 93\% reduction in chlorine usage.

\TABLE{cccc}
{
	\small
	Year & Treatment & Surviving & Chlorine \\
	& duration & mussels & consumption \\
	& (hour) & (\%) & (kg) \\
	\hline
	2021 & 23-0 & 8.1 & 367 \\
	2021 & 22-0 & 7.0 & 704 \\
	2021 & 20-23 & 6.5 & 860 \\
	2021 & 18-0 & 5.8 & 1,517 \\
	2021 & 15-3 & 4.9 & 2,999 \\
	\hline
	2022 & 0-1 & 9.4 & 372 \\
	2022 & 23-1 & 7.8 & 721 \\
	2022 & 23-2 & 7.1 & 1,088 \\
	2022 & 21-3 & 6.0 & 2,042 \\
	2022 & 20-8 & 5.1 & 3,711
}{Percentage of surviving settled mussels and chlorine consumptions for the
optimal intermittent treatments in the CS-WUA \#1 network for the 2021 and 2022
irrigation seasons}{TabMClOptimal}

In Figure~\ref{FigMCl}, the percentage of surviving settled mussels versus free
chlorine consumed in the intermittent and continuous treatments for the 2021 and
2022 irrigation campaigns is depicted. Generally, higher chlorine consumption
results in higher mussel mitigation. However, a curious pattern similar to
hysteresis cycles is observed in intermittent treatments because, with the same
amounts of chlorine, the treatment is more effective when applied at moments
before the peak discharge than when applied afterwards. This effect is more
pronounced in short-duration treatments. 

\FIGII{Figure_9a.pdf}{Figure_9b.pdf}
{Percentage of surviving settled mussels versus the consumption of free chlorine
with different chlorine treatments, continuous and intermittent with different
duration and starting times, in the 2021 (left) and 2022 (right) irrigation
seasons}{FigMCl}

\subsection{Risk of on-farm irrigation system infestation}

In a third simulation analysis, we addressed the efect of the chlorine
treatments on the number of larvae and free chlorine mass discharged through 
network hydrants to the on-farm irrigation systems. This process is an indicator
of the capacity of the network to transfer the infestation risk and the chemical
protection, respectively, to the downstream hydraulic structures.

In Figure~\ref{FigExportLarvae}, the simulated quantities of mussel larvae
exported by the different hydrants in the network during the 2022 irrigation
campaign are shown for different chlorine treatments. The total number of
exported larvae ($N_e$) is displayed above each subfigure. Without treatment,
the number of larvae exported by each hydrant is directly proportional to its
demand. It can be seen that both the sublethal dose treatment and the
intermittent treatment barely reduce the number of exported larvae, which could
result in the infestation of the on-farm irrigation networks. In contrast, the
continuous treatment with double the lethal dose results in a very strong
reduction in exported larvae. The hydrants closest to the inlet have the highest
exports because the larvae have had less contact time with the free chlorine.

\begin{figure}[ht!]
	\centering
	\begin{tabular}{cc}
		$N_e=5.27\cdot10^7$~larvae &
		$N_e=2.66\cdot10^8$~larvae \\
		\includegraphics[width=0.34\textwidth]
		{Figure_10a.pdf} &
		\includegraphics[width=0.34\textwidth]
		{Figure_10b.pdf} \\
		$N_e=1.08\cdot10^9$~larvae &
		$N_e=1.45\cdot10^9$~larvae \\
		\includegraphics[width=0.34\textwidth]
		{Figure_10c.pdf} &
		\includegraphics[width=0.34\textwidth]
		{Figure_10d.pdf} \\
	\end{tabular}
	$N_e=1.56\cdot10^9$~larvae \\
	\includegraphics[width=0.34\textwidth]
	{Figure_10e.pdf}
	\caption{Maps showing the simulated quantities of mussel larvae exported
		by the hydrants in the CS-WUA~\#1 network for the 2022
		irrigation campaign: with free chlorine continuous treatments at
		concentration of $1.0\cdot10^{-3}$~kg$\cdot$m$^{-3}$ (top-left),
		$0.5\cdot10^{-3}$~kg$\cdot$m$^{-3}$ (top-right), and
		$0.1\cdot10^{-3}$~kg$\cdot$m$^{-3}$ (middle-left); with
		intermittent free chlorine treatment of 1-hour duration starting
		at 0:00 hours at a concentration of
		$1.0\cdot10^{-3}$~kg$\cdot$m$^{-3}$ (middle-right); and without
		treatment (bottom). The circle areas are proportional to the
		number of larvae exported by each hydrant. The total number of
		exported larvae is displayed above each subfigure.
		\label{FigExportLarvae}}
\end{figure}

Figure~\ref{FigExportChlorine} displays the simulated masses of free chlorine
exported by the network hydrants for the same treatments. It is observed that
the hydrants closest to the inlet have the highest exports, which could prevent
the infestation of on-farm irrigation structures and at the same time cause
environmental concerns. The more distant hydrants exhibit a greater decline in
concentration, resulting in lower exports. Continuous sublethal or intermittent
treatments show a strong reduction in the total mass of exported chlorine,
although there is little variation in its spatial distribution. The ratio of the
total exported mass of free chlorine to the injected mass ranged from 12\% to
20\%.

\FIGIVB{Figure_11a.pdf}{Cl=651~kg}
{Figure_11b.pdf}{Cl=340~kg}
{Figure_11c.pdf}{Cl=65~kg}
{Figure_11d.pdf}{Cl=74~kg}
{Maps showing the simulated masses of free chlorine exported by the hydrants in
the CS-WUA~\#1 network for the 2022 irrigation campaign: with continuous
treatments at concentrations of $1.0\cdot10^{-3}$~kg$\cdot$m$^{-3}$ (top-left),
$0.5\cdot10^{-3}$~kg$\cdot$m$^{-3}$ (top-right),  and
$0.1\cdot10^{-3}$~kg$\cdot$m$^{-3}$ (bottom-left); and with intermittent
chlorine treatment of 1 hour duration starting at 0:00 hours at a
concentration of $1.0\cdot10^{-3}$~kg$\cdot$m$^{-3}$ (bottom-right). The area of
the circles is proportional to the free chlorine mass exported by each hydrant.
The total mass of exported free chlorine is displayed above each subfigure
\label{FigExportChlorine}}

\subsection{Discusion}

The proposed model has expanded the concepts and processes described by
\citet{JinZhao21} from rivers to the geometry and the managements habits of
collective pressurized irrigation networks. The inclusion of chemical treatments
required major modifications to the model.

The high observed variability in larval entry does not produce relevant changes
in the amount and in the spatial distribution of settled mussels in the pipe
network. Mussels settlement seems to be primarly associated to the hydraulic
network morphology. Thus, similar results are obtained in simulations with
different seasonal entry patterns of the same number of larvae. However, both
the total number of larvae and their settlement rate have proven to be
determinant in network infestation. Since the relevant input parameters are
estimated with substantial uncertainty, our model will also have considerable
uncertainty. Therefore, we should interpret its results more qualitatively than
quantitatively.

Only continuous free chlorine treatments with concentrations at or above the
lethal larval level have achieved more than 99\% reduction in zebra mussel
infestation. However, the required free chlorine quantities (up to
3.5~kg$\cdot$ha$^{-1}$ $\approx$ 25~\euro{}$\cdot$ha$^{-1}$)  may be
economically and environmentally unsustainable.

Interruptions in treatment lead to live larvae reaching distant areas from the
inlet, where free chlorine does not have sufficient concentration to kill them.
Daily short-duration treatments (from 1 to 3 hours) have been more effective in
mussel removal when used just before the daily network peak discharge. We
believe this is because flow velocity is higher at the peak, allowing chlorine
to reach distant areas from the inlet with higher concentration. Concentrating
chlorine injection during these moments has shown potential for achieving up to
93\% chlorine savings, while retaining mussel setlement mitigation rates above
90\%. However, continuous treatments with larval sublethal doses have proven to
be more effective, achieving a greater reduction in infestation for the same
total amounts of chlorine compared to intermittent treatments. The continuous
treatment with a dose at 10\% of the lethal larval level also resulted in over
90\% infestation mitigation while saving 95\% of the injected chlorine.

The proposed model permits to assess the effectiveness of different strategies
for chemical treatment aiming at the protection of the pressurized collective
network from colonization by adults, flow obstruction and ultimately from
becoming a secondary source of larvae. We believe that this model can be used in
conjunction with the Normalized Pressure Method \citep{MoralesHernandez18,
MoralesHernandez22} for advanced control of the infestation, combining
capacities for early detection and targeted, optimized chemical treatment.

The larvae exported through the network hydrants can pose a significant
problem for on-farm irrigation networks. Among the simulated cases, only
continuous free chlorine injection at double the lethal larval concentration
achieves a strong reduction (97\%) in larval export. Continuous injection at the
lethal dose reduced the export by 84\%. With sublethal doses or intermittent
treatments, a substantial percentage of larvae are still exported through the
hydrants.

Although most of the free chlorine (80\% to 88\%) is neutralized within the
network before being exported through the hydrants, the amounts of chlorine that
do get exported, particularly in hydrants close to the injection point, could
have an environmental impact. With intermittent injection treatments, an
operational strategy of avoiding the opening of the nearest hydrants during
injection periods could help to reduce the problem. This would also increase the
effectiveness of the treatment.

In this paper, the proposed model has been used to gain insight on zebra mussel
infestation and chlorine treatment in collective pressurized irrigation
networks. However, the model can be parametrized to simulate other invasive
species, different chemicals and other types of pressurized networks.

\section{Conclusion}

The proposed model is a valuable tool for identifying infestation zones and
assessing chemical treatments for mussel control in pressurized networks. It
shows that variability in larval entry has minimal impact on mussel settlement,
which is more influenced by network morphology. Continuous chlorine treatments
at or above lethal concentrations reduce infestation by over 99\%, but may be
economically and environmentally unsustainable. Intermittent treatments are less
effective, allowing larvae to reach distant areas, while short-duration
treatments before peak discharge can save up to 93\% of chlorine and still
achieve over 90\% infestation reduction. Continuous sublethal doses are more
effective, achieving over 90\% mitigation with significant chlorine savings. The
model helps evaluate treatment strategies to protect networks from colonization
and larval export, which can pose problems for on-farm irrigation. Although most
free chlorine is neutralized within the network, exported chlorine could impact
the environment. Avoiding the opening of nearby hydrants during intermittent
treatments can mitigate this issue. The model has many sources of uncertainty.
Results should therefore be considered more qualitatively than quantitatively.

\section*{Aknowledgments}

Thanks are due to the managers of the Collarada Segunda Water Users Association.
This research would not be possible without their cooperation.

Funding: this research was funded through:
\begin{itemize}
\item Grant PID2021-124095OB-I00 by MCIN/AEI/10.13039/501100011033 and by ERDF,
	a way of making Europe.
\item Grant GCP2022000400 by the Livestock and Environment of the Regional
	Government of Aragón and the European Agricultural Fund for Rural
	Development (EAFRD) of the European Union through the Rural Development
	Programme.
\end{itemize}

\appendix
\section{Supporting information}
Supplementary data associated with this article can be found in the online
version at \url{doi:}

\bibliographystyle{elsarticle-harv}
\bibliography{bib}

\end{document}
