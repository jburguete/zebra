\documentclass[review,authoryear]{elsarticle}

\usepackage[utf8]{inputenc}
\usepackage[english]{babel}

\title{A coupled model of transport and decay of solutes and mussel larvae in
	pressurized networks}

\author[1]{Javier Burguete}
\ead{jburguete@eead.csic.es}

\author[1]{Borja Latorre}

\author[1]{Pilar Paniagua}

\author[1]{Eva Medina}

\author[1]{Javier Fernández-Pato}

\author[1]{Enrique Playán}

\author[1]{Nery Zapata}

\affiliation[1]{organization={Soil and Water, EEAD/CSIC},
	addressline={Av. Montañana 1005},
	postcode={50059},
	city={Zaragoza},
	country={Spain}}

\date{\today}

\newcommand{\EQ}[2]{\begin{equation}#1\label{#2}\end{equation}}
\newcommand{\PA}[1]{\left(#1\right)}
\newcommand{\PARTIAL}[2]{\frac{\partial#1}{\partial#2}}

\begin{document}

\begin{abstract}
In this work we present ...
\end{abstract}

\begin{highlights}
\item 1st highlight
\end{highlights}

\begin{keyword}
mussel larvae, solute transport, solute decay, pressurized networks, irrigation
\end{keyword}

\maketitle

\section*{Notation}

\begin{tabular}{lcl}
$A$ & = & cross sectional area,\\
$c$ & = & solute (or mussel larvae) concentration,\\
$f$ & = & friction factor of Darcy-Weissbach,\\
$K_x$ & = & longitudinal dispersion coefficient,\\
$P$ & = & perimeter,\\
$Q$ & = & discharge,\\
$S$ & = & solute (or mussel larvae) source,\\
$t$ & = & time,\\
$v$ & = & velocity,\\
$x$ & = & longitudinal coordinate,\\
\end{tabular}

\section{Introduction}

\section{Theorical models}

\subsection{Model of transport and dispersion of solutes in a pipe}

The equation governing a one-dimensional transport and dispersion of solutes in
a pipe is the advection-diffusion Fick's law with sources due to external
injections or reaction decays:
\EQ{\PARTIAL{}{t}(A\,c)+\PARTIAL{}{x}(Q\,c)
	=S+\PARTIAL{}{x}\PA{K_x\,A\,\PARTIAL{c}{x}}}{EqTransport}
where $t$ is the time, $A$ the cross sectional area of the pipe, $c$ the solute
concentration, $x$ the longitudinal coordinate, $Q$ the discharge, $S$ the
solute source and $K_x$ the longitudinal dispersion coefficient.

In this work, the following model, according to the \citet{Rutherford94} works,
for $K_x$ coefficient is used:
\EQ{K_x=10\,P\,|v|\,\sqrt{\frac{f}{2}}}{EqKx}  
with $P$ the perimeter of the pipe, $v$ the water velocity and $f$ friction
factor of Darcy-Weissbach \citep{Darcy58}.

\subsection{Model of solute decay in a pipe}

\subsection{Model of transport and dispersion of mussel larvae in a pipe}

The swimming capacity of the zebra mussel larvae can be considered negligible
compared to the water velocity in pressurized networks. Then, equation
\ref{EqTransport} is used for transport and dispersion of larvae in pipes.

\subsection{Model of mussel larvae cling in a pipe}

\subsection{Model of mussel larvae death}

\subsection{Model of junctions}

\subsection{Model of inlets}

\section{Applications cases}

\section{Conclusions}

\section*{Aknowledgments}

\bibliographystyle{elsarticle-harv}
\bibliography{bib}

\end{document}
