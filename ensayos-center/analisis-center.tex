\documentclass[a4paper]{article}

\author{Javier Burguete, Javier Fernández-Pato, Borja Latorre, Nery Zapata}

\date{\today}

\title{Análisis adimensional del rozamiento en una tubería con tornillos}

\newcommand{\EQ}[2]{\begin{equation}#1\label{#2}\end{equation}}

\begin{document}

\section{Fuerzas de rozamiento en flujo turbulento}

\subsection{Cilindro en flujo uniforme}

El rozamiento que produce un cilindro, colocado transversalmente en un flujo
uniforme turbulento es:
\EQ{F_R=\frac12\,c_c'\,\rho\,u^2\,D_c\,L_c}{EqFrCilindroUniforme}
con $c_c'$ el coeficiente de resistencia aerodinámica del cilindro, $\rho$ la
densidad del fluido, $u$ la velocidad no perturbada del fluido, $D_c$ el
diámetro del cilindro y $L_c$ la longitud del cilindro. Para altos números de
Reynolds la literatura propone un valor de este coeficiente de:
\EQ{c_c'=0.51}{EqCc}

\subsection{Cilindro en una pared}

La fórmula anterior cambia si el cilindro está colocado sobre una pared debido
al perfil de velocidades vertical sobre la pared:
\EQ{F_R=\frac12\,c_c\,\rho\,u_*^2\,D_c\,L_c}{EqFrCilindroPared}
con $c_c$ un nuevo coeficiente aerodinámico del cilindro en la pared y $u_*$ la
velocidad del fluido a la distancia $L_c$ a la pared:
\EQ{u_*=u\left(z=L_t\right)}{EqUAsterisco}
La velocidad $u_*$ es la máxima que se enfrenta al tornillo, pero en distancias
más cercanas a la pared la velocidad del fluido va progresivamente disminuyendo
hasta anularse, por tanto la resistencia es menor que en el caso anterior y es
esperable:
\EQ{c_c<c_c'}{EqCcII}

\subsection{Conjunto de tornillos en la pared de una tubería}

En una tubería prismática de sección de circular de longitud $L$ y diámetro $D$
sobre la que se colocan $N_t$ tornillos iguales en la pared, haciendo la
aproximación de los tornillos como cilindros, la fuerza que hace el flujo
sobre los tornillos será:
\EQ{F_R=\frac12\,N_t\,c_t\,\rho\,u_*^2\,D_t\,L_t}{EqFrTornillos}
con $c_t$, $D_t$ y $L_t$ el coeficiente de resistencia aerodinámico en la pared,
el diámetro y la longitud respectivamente de los tornillos.

Considerando que la fuerza de los tornillos es muy superior a la fuerza de
rozamiento viscosa sobre la superficie lisa de la tubería, la pérdida de presión
$\Delta P$ que experimentará la tubería será:
\[
  \Delta P\,A=-F_R\Rightarrow\quad
  \Delta P\,\frac{\pi\,D^2}{4}=-\frac12\,N\,c_t\,\rho\,u_*^2\,D_t\,L_t
  \Rightarrow
\]
\EQ{\Delta P=-\frac{2\,N_t\,c_t\,\rho\,u_*^2\,D_t\,L_t}{\pi\,D^2}}
{EqDPTornillos}
con $A$ el área de la sección. El número de tornillos puede expresarse como una
densidad superficial de tornillos $\sigma_t$ por la superficie lateral de la
tubería:
\EQ{N_t=\sigma_t\,\pi\,D\,L}{EqDensidadTornillos}
Sustituyendo en (\ref{EqDPTornillos}):
\EQ{\Delta P=-\frac{2\,\sigma_t\,c_t\,\rho\,u_*^2\,D_t\,L_t\,L}{D}}
{EqPerdidaTornillos}

\section{Leyes de rozamiento en tuberías}

\subsection{Darcy-Weisbach}

La ley de rozamiento en tuberías de Darcy-Weisbach es:
\EQ{\Delta P=-f\,\frac12\,\rho\,U^2\,\frac{L}{D}}{EqDarcyWeisbach}
con $f$ el factor de fricción de Darcy-Weisbach (adimensional) y $U$ la
velocidad media del flujo.

Sustituyendo en (\ref{EqPerdidaTornillos}) obtenemos el factor de fricción para
la tubería con las paredes con tornillos:
\EQ{f=4\,\sigma_t\,c_t\,D_t\,L_t\,\left(\frac{u_*}{U}\right)^2}
{EqDarcyWeisbachTornillos}

La relación $\frac{u_*}{U}$ depende exclusivamente del perfil de velocidades en
la tubería. En coordenadas polares, colocando el origen en el centro de la
tubería:
\[u_*=u\left(r=R-L_t\right),\]
\EQ
{
  U=\frac{1}{A}\int_Au\,dA=\frac1{\pi\,R^2}\,\int_0^{2\,\pi}d\theta\,
  \int_0^Ru(r)\,r\,dr=\frac2{R^2}\,\int_0^Ru(r)\,r\,dr
}{EqU}
con $R=\frac{D}{2}$ el radio de la tubería.

\subsubsection{Perfil de velocidades exponencial}

Para un perfil de velocidades exponencial:
\[u(r)=u_*\,\left(\frac{R-r}{L_t}\right)^a\Rightarrow\]
\EQ
{
  U=\frac2{R^2}\,\int_0^Ru_*\,\left(\frac{R-r}{L_t}\right)^a\,r\,dr
  =\frac2{(1+a)\,(2+a)}\,u_*\,\left(\frac{R}{L_t}\right)^a
}{EqUExponencial}
con $a$ el exponente de ajuste. Sustituyendo en
(\ref{EqDarcyWeisbachTornillos}):
\EQ
{
  f=2^{2\,a}\,(1+a)^2\,(2+a)^2\,\sigma_t\,c_t\,D_t\,L_t
  \,\left(\frac{L_t}{D}\right)^{2\,a}
}{EqfExponencial}
Nótese que, de acuerdo con esta fórmula, para un perfil exponencial el factor de
fricción de Darcy-Weisbach es proporcional a los siguientes factores:
\EQ
{
  f\;\alpha\;\sigma_t,\quad
  f\;\alpha\;D_t,\quad
  f\;\alpha\;L_t^{1+2\,a},\quad
  f\;\alpha\;D^{-2\,a}
}{EqfExponencialFactores}

\subsubsection{Perfil de velocidades logarítmico}

Para un perfil de velocidades logarítmico, según la hipótesis de Kármán-Prandtl:
\[
  u(r)=\frac{u_*}{\kappa}
  \,\log\left(\frac{R-r}{z_0}\right),\;R-r\geq z_0\Rightarrow\quad
  u_*=\frac{u_*}{\kappa}\,\log\left(\frac{L_t}{z_0}\right),\]
\EQ
{
  U=\frac2{R^2}\,\int_0^{R-z_0}
  \frac{u_*}{\kappa}\,\log\left(\frac{R-r}{z_0}\right)\,r\,dr
  =\frac{u_*}{\kappa}\,\left[\log\left(\frac{R}{z_0}\right)-\frac32
  +\frac{2\,z_0\,\left(R-\frac14\,z_0\right)}{R^2}\right]
}{EqULogaritmico}
con $\kappa=0.4$ la constante de Kármán-Prandtl y $z_0$ una longitud
característica de rugosidad. En caso de bajas rugosidades relativas:
\EQ
{
  R\gg z_0\Rightarrow\;
  U\approx\frac{u_*}{\kappa}
  \,\left[\log\left(\frac{R}{z_0}\right)-\frac32\right]
}{EqULogaritmicoAproximado}
Sustituyendo en (\ref{EqDarcyWeisbachTornillos}) y manipulando logaritmos:
\EQ
{
  f=4\,\sigma_t\,c_t\,D_t\,L_t\,\left[\frac{\log\left(\frac{L_t}{z_0}\right)}
  {\log\left(\frac{R}{z_0}\right)-\frac32}\right]^2
  =\frac1{\log^2\left[\left(2\,e^{\frac32}\,\frac{z_0}{D}\right)^{\frac1{2\,
  \sqrt{\sigma_t\,c_t\,D_t\,L_t}\,\log\left(\frac{z_0}{L_t}\right)}}\right]}
}{EqfLogaritmico}

\subsection{Gauckler-Manning}

La ley de rozamiento de Gauckler-Manning es:
\EQ{\Delta P=-g\,n^2\,\rho\,U^2\,\frac{L}{R_h^{\frac43}}}{EqGaucklerManning}
con $n$ el coeficiente de rozamiento de Gauckler-Manning, $g$ la aceleración
gravitatoria y $R_h$ el radio hidráulico:
\EQ{R_h=\frac{A}{P_r}=\frac{D}{4}}{EqRadioHidraulico}
con $A$ y $P_r$ el área y el perímetro respectivamente de la sección. Esta ley
es equivalente a la de Darcy-Weisbach (\ref{EqDarcyWeisbach}) si:
\EQ{f=\frac{2^{\frac{11}{3}}\,g\,n^2}{D^{\frac13}}}{EqDarcyWeisbachManning}
Comparando con (\ref{EqfExponencial}) se comprueba que esta ecuación es
equivalente a la fuerza de resistencia de los tornillos con un perfil
exponencial si:
\EQ
{
  a=\frac16,\quad
  n^2=\frac{\left(\frac{91}{36}\right)^2\,\sigma_t\,c_t\,D_t\,L_t^{\frac43}}
  {2^{\frac{10}{3}}\,g},\quad
  f=2^{\frac13}\,\left(\frac{91}{36}\right)^2\,\sigma_t\,c_t\,D_t\,L_t
  \,\left(\frac{L_t}{D}\right)^{\frac13}
}{EqManningExponencial}
Por tanto, si damos como buena la ley de Gauckler-Manning, es esperable de
que el factor de fricción de Darcy-Weisbach sea proporcional a:
\EQ
{
  f\;\alpha\;\sigma_t,\quad
  f\;\alpha\;D_t,\quad
  f\;\alpha\;L_t^{\frac43},\quad
  f\;\alpha\;D^{-\frac13}
}{EqfExponencialFactoresManning}
También es posible despejar el coeficiente de fricción de los tornillos en la
pared en (\ref{EqManningExponencial}) a partir de medidas experimentales de $f$:
\EQ
{
  c_t=2^{-\frac13}\,\left(\frac{36}{91}\right)^2\,\frac{f}{\sigma_t\,D_t\,L_t}
  \,\left(\frac{D}{L_t}\right)^{\frac13}
}{EqManningct}

\subsection{Colebrook-White}

La expresión de la fórmula de Colebrook-White para el factor de fricción de
Darcy-Weisbach es la siguiente:
\EQ
{
  \frac1{\sqrt{f}}=-2\,\log_{10}\left(\frac{k_s}{3.7\,D}
  +\frac{2.51}{\mathrm{Re}\,\sqrt{f}}\right)
}{EqColebrookWhite}
con Re el número de Reynolds y $k_s$ una longitud característica de rugosidad.
En medios como los que nos ocupan con rugosidad relativa relativamente alta
debida a la presencia de los tornillos y altos números de Reynolds el valor del
segundo sumando es mucho menor que el del primero y la expresión anterior se
reduce a la segunda ley de Kármán-Prandtl:
\EQ{\frac1{\sqrt{f}}=-2\,\log_{10}\left(\frac{k_s}{3.7\,D}\right)}
{EqKarmanPrandtl}
Despejando y manipulando logaritmos:
\EQ
{
  f=\frac{\log^2 10}{4\,\log^2\left(\frac{k_s}{3.7\,D}\right)}
  =\frac1{\log^2\left[\left(\frac{k_s}{3.7\,D}\right)^{\frac2{\log 10}}\right]}
}{EqfKarmanPrandtl}
Comparando con ($\ref{EqfLogaritmico}$) esta ecuación es igual si:
\EQ
{
  \frac{k_s}{3.7}=2\,e^{\frac32}\,z_0,\quad
  \frac{2}{\log 10}
  =\frac1{2\,\sqrt{\sigma_t\,c_t\,D_t\,L_t}\,\log\left(\frac{z_0}{L_t}\right)}
}{EqRugosidadesI}
de donde:
\EQ
{
  z_0=L_t\,\exp\left(\frac{\log 10}{4\,\sqrt{\sigma_t\,c_t\,D_t\,L_t}}\right),
  \quad
  k_s=7.4\,e^{\frac32}\,L_t
  \,\exp\left(\frac{\log 10}{4\,\sqrt{\sigma_t\,c_t\,D_t\,L_t}}\right)
}{EqRugosidadesII}
y sustituyendo en (\ref{EqfKarmanPrandtl}) se obtiene una complicada relación
entre el factor de fricción de Darcy-Weisbach y el número y dimensiones de
tornillos y el diámetro de la tubería:
\[
  f=\frac{\log^2 10}{4\,\log^2\left[\frac{2\,e^{\frac32}\,L_t}{D}
  \,\exp\left(\frac{\log 10}{4\,\sqrt{\sigma_t\,c_t\,D_t\,L_t}}\right)\right]}
  =\frac1{4\,\left[\log_{10}\left(\frac{2\,e^{\frac32}\,L_t}{D}
  \,10^{\frac1{4\,\sqrt{\sigma_t\,c_t\,D_t\,L_t}}}\right)\right]^2}
\]
\EQ
{
  =\frac1{4\,\left[\frac1{4\,\sqrt{\sigma_t\,c_t\,D_t\,L_t}}
  +\log_{10}\left(\frac{2\,e^{\frac32}\,L_t}{D}\right)\right]^2}
  =\frac{4\,\sigma_t\,c_t\,D_t\,L_t}{\left[1+4\,\sqrt{\sigma_t\,c_t\,D_t\,L_t}
  \,\log_{10}\left(\frac{2\,e^{\frac32}\,L_t}{D}\right)\right]^2}
}{EqfKarmanPrandtlFactors}
También aquí se puede obtener $c_t$ a partir de medidas experimentales de $f$.
haciendo la raíz cuadrada:
\EQ
{
  \sqrt{f}=\frac{2\,\sqrt{\sigma_t\,c_t\,D_t\,L_t}}
  {\left|1+4\,\sqrt{\sigma_t\,c_t\,D_t\,L_t}
  \,\log_{10}\left(\frac{2\,e^{\frac32}\,L_t}{D}\right)\right|}
}{EqfKarmanPrandtlFactorsRoot}
y definiendo el signo:
\EQ
{
  s=\frac{\left|1+4\,\sqrt{\sigma_t\,c_t\,D_t\,L_t}
  \,\log_{10}\left(\frac{2\,e^{\frac32}\,L_t}{D}\right)\right|}
  {1+4\,\sqrt{\sigma_t\,c_t\,D_t\,L_t}
  \,\log_{10}\left(\frac{2\,e^{\frac32}\,L_t}{D}\right)}
}{EqfKarmanPrandtlFactorsSign}
se puede despejar en (\ref{EqfKarmanPrandtlFactorsRoot}) finalmente $c_t$:
\EQ
{
  c_t=\frac14\,\frac{f}{\sigma_t\,D_t\,L_t}\,\frac1{\left[
  1-2\,s\,\sqrt{f}\,\log_{10}\left(\frac{2\,e^{\frac32}\,L_t}{D}\right)\right]^2}
}{EqfKarmanPrandtlct}

\end{document}
