\documentclass[a4paper]{article}

\usepackage[utf8]{inputenc}
\usepackage{graphicx}
\usepackage[spanish]{babel}

\title{Ensayos del CENTER}

\author{J. Burguete, N. Zapata, E. Playán, ...}

\date{\today}

\begin{document}

\maketitle

\newcommand{\EQ}[1]{\begin{equation}#1\end{equation}}
\newcommand{\FIG}[3]
{
  \begin{figure}[ht!]
    \centering
    \includegraphics{#1}
    \caption{#2\label{#3}}
  \end{figure}
}
\newcommand{\FIGIV}[6]
{
  \begin{figure}[ht!]
    \centering
    \begin{tabular}{cc}
      \includegraphics{#1} & \includegraphics{#2}\\
      \includegraphics{#3} & \includegraphics{#4}
    \end{tabular}
    \caption{#5\label{#6}}
  \end{figure}
}
\newcommand{\PA}[1]{\left(#1\right)}
\newcommand{\PARTIAL}[2]{\frac{\partial#1}{\partial#2}}

\section{Análisis adimensional}

En una tubería prismática de longitud infinita, para flujo incompresible y
estacionario, las ecuaciones de Navier-Stokes que rigen el movimiento del fluido
se reducen a:
\[Q=\mathrm{const.}\]
\EQ{\frac1{\rho}\,\PARTIAL{p}{x}=g\,\PA{S_0-S_f}}
con $Q$ el caudal, $\rho$ la densidad del fluido, $p$ la presión, $g$
la constante de gravedad, $S_0$ la pendiente y $S_f$ la pendiente asociada al
esfuerzo de rozamiento. Expresando esta última ecuación en función del esfuerzo
de rozamiento $\tau_f$: 
\EQ{\PARTIAL{p}{x}=g\,\rho\,S_0-\frac{P\,\tau_f}{A}}
con $P$ el perímetro y $A$ el área de sección de la tubería. Si la tubería es
horizontal $\PA{S_0=0}$, la única causa de pérdida de carga es el rozamiento:
\EQ{\PARTIAL{p}{x}=-\frac{P\,\tau_f}{A}}
que en tuberías de sección circular de diámetro $D$ es:
\EQ{\PARTIAL{p}{x}=-\frac{4\,\tau_f}{D}}

Uno de los modelos más antiguos para expresar el esfuerzo de fricción en
formulación adimensional es el modelo de Darcy-Weisbach, que expresa este
esfuerzo como un factor por la presión de ariete:
\EQ{\tau_f=f\,\frac12\,\rho\,U^2}
con $U$ la velocidad media y $f$ el factor de fricción adimensional de
Darcy-Weisbach. Aplicando el modelo a la pérdida de carga:
\EQ{\PARTIAL{p}{x}=-\frac{2\,f\,\rho\,U^2}{D}}
En una tubería de longitud $L$ la pérdida de carga es:
\EQ{\Delta p=\int_0^L\PARTIAL{p}{x}\,dx=L\,\PARTIAL{p}{x}}
de donde se puede extraer el factor de fricción a partir de la pérdida de carga:
\EQ{f=-\frac{\Delta p\,D}{2\,\rho\,U^2\,L}}
En principio $f$ depende de la velocidad, el diámetro, la viscosidad del fluido
$(\nu)$ y la rugosidad del material, que caracterizaremos con una longitud
característica $l$. Aplicando el teorema Pi esta función dependerá únicamente de
2 números adimensionales:
\EQ
{
  f=f(D,\,U,\;\nu,\;l)=f(\mathrm{Rl},\;\mathrm{Re}),\quad
  \mathrm{Rl}=\frac{l}{D},\quad
  \mathrm{Re}=\frac{U\,D}{\nu}
}
con Rl la rugosidad relativa y Re el número de Reynolds adimensionales.

Para flujos laminares (Re $<2000$) el factor de fricción sigue la fórmula de
Poiseuille:
\EQ{\mathrm{Re}<2000\Rightarrow\;f=\frac{64}{\mathrm{Re}}}
Para flujos netamente turbulentos (Re $>4000$) la fórmula más usada es la de
Colebrook-White:
\EQ
{
  \mathrm{Re}>4000\Rightarrow\;
  \frac1{\sqrt{f}}=-2\,\log_{10}\PA{\frac{\mathrm{Rl}}{3.7}
  +\frac{2.51}{\mathrm{Re}\,\sqrt{f}}}
}\label{EqColebrookWhite}
que en caso de rugosidades relativas altas y números de Reynolds altos puede
simplificarse puesto que el primer sumando de la fórmula de Colebrook-White es
mucho mayor que el segundo:
\[
  \mathrm{Rl,\,Re}\,\uparrow\uparrow\,\Rightarrow\;
  \frac1{\sqrt{f}}\approx-2\,\log_{10}\PA{\frac{\mathrm{Rl}}{3.7}}
  =2\,\PA{\log_{10}3.7-\log_{10}\mathrm{Rl}}\Rightarrow\;
\]
\EQ
{
  f\approx\frac1{4\PA{\log_{10}3.7-\log_{10}\mathrm{Rl}}^2}
  \label{EqColebrookWhiteSimplified}
}
En esta aproximación también puede obtenerse la rugosidad relativa a partir
del factor de fricción:
\EQ{\mathrm{Rl}=3.7\,\cdot\,10^{-0.5/\sqrt{f}}\label{EqRl}}

\section{Ensayos realizados}

En la \figurename~\ref{FigRlRe} presentamos todo el espacio barrido por los
experimentos en los números adimensionales Rl y Re. Puede verse que se ha hecho
un recorrido bastante exhaustivo de ambos parámetros en los números que suelen
caracterizar las tuberías de riego por aspersión y de las rugosidades asociadas
a presencia de mejillón cebra.
\FIG{rl-re.eps}{Espacio barrido experimentalmente en Re y Rl.}{FigRlRe}

En las \figurename{s}~\ref{Figf0}-\ref{Figf45dn300} se representa el factor $f$
de Darcy-Weisbach en todos los experimentos realizados. De acuerdo a la fórmula
(\ref{EqColebrookWhiteSimplified}) este factor debería ser aproximadamente
constante en flujos netamente turbulentos y rugosidades altas, lo que sucede en
todos experimentos realizados con tornillos (la aproximación en cambio no es
aplicable al caso con tuberías lisas). En la tubería lisa $f$ parece tener un
comportamiento aproximande constante, con tendencia a disminuir a mayores velocidades, para DN200. Para DN250 el comportamiento es más errático, probablemente
porque las pérdidas son tan bajas que entramos en problemas de precisión de la
medida. Para los casos con tornillos $f$ ha resultado bastante constante en
general, para misma longitud y densidad de tornillos, aunque alguna serie de
experimentos muestra también un comportamiento algo errático, sobre todo con los
caudales más bajos, por lo que creemos que pudieron tener algún fallo en la
medida de presión.
\FIG{f0.eps}{$f$ en función de $H_f$ para tuberías sin tornillos.}{Figf0}
\FIG{f20-dn200.eps}{$f$ en función de $H_f$ para tuberías con tornillos a 20 mm
para DN 200.}{Figf20dn200}
\FIG{f20-dn250.eps}{$f$ en función de $H_f$ para tuberías con tornillos a 20 mm
para DN 250.}{Figf20dn250}
\FIG{f20-dn300.eps}{$f$ en función de $H_f$ para tuberías con tornillos a 20 mm
para DN 300.}{Figf20dn300}
\FIG{f25-dn300.eps}{$f$ en función de $H_f$ para tuberías con tornillos a 25 mm
para DN 300.}{Figf25dn300}
\FIG{f30-dn200.eps}{$f$ en función de $H_f$ para tuberías con tornillos a 30 mm
para DN 200.}{Figf30dn200}
\FIG{f30-dn250.eps}{$f$ en función de $H_f$ para tuberías con tornillos a 30 mm
para DN 250.}{Figf30dn250}
\FIG{f30-dn300.eps}{$f$ en función de $H_f$ para tuberías con tornillos a 30 mm
para DN 300.}{Figf30dn300}
\FIG{f35-dn300.eps}{$f$ en función de $H_f$ para tuberías con tornillos a 35 mm
para DN 300.}{Figf35dn300}
\FIG{f40-dn200.eps}{$f$ en función de $H_f$ para tuberías con tornillos a 40 mm
para DN 200.}{Figf40dn200}
\FIG{f40-dn250.eps}{$f$ en función de $H_f$ para tuberías con tornillos a 40 mm
para DN 250.}{Figf40dn250}
\FIG{f40-dn300.eps}{$f$ en función de $H_f$ para tuberías con tornillos a 40 mm
para DN 300.}{Figf40dn300}
\FIG{f45-dn300.eps}{$f$ en función de $H_f$ para tuberías con tornillos a 45 mm
para DN 300.}{Figf45dn300}

Finalmente, en las \figurename{s}~\ref{Figl20}-\ref{Figl45} se representa la
longitud característica de rugosidad obtenida a partir de $f$ mediante la
aproximación (\ref{EqRl}) y se compara frente a la longitud de los tornillos.
Salvo algún caso con comportamiento errático, en general se obtiene una
longitud de rugosidad característica inferior a la longitud de los tornillos,
siendo menor cuando se disminuyó la densidad.
\FIG{l20.eps}{$l$ en función de $H_f$ para tuberías con tornillos a 20 mm.}
{Figl20}
\FIG{l25.eps}{$l$ en función de $H_f$ para tuberías con tornillos a 25 mm.}
{Figl25}
\FIG{l30.eps}{$l$ en función de $H_f$ para tuberías con tornillos a 30 mm.}
{Figl30}
\FIG{l35.eps}{$l$ en función de $H_f$ para tuberías con tornillos a 35 mm.}
{Figl35}
\FIG{l40.eps}{$l$ en función de $H_f$ para tuberías con tornillos a 40 mm.}
{Figl40}
\FIG{l45.eps}{$l$ en función de $H_f$ para tuberías con tornillos a 45 mm.}
{Figl45}

\section{Conclusiones}

\end{document}
