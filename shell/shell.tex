\documentclass[a4paper]{article}

\usepackage{graphicx}
\usepackage[spanish]{babel}

\author{Javier Burguete}
\date{\today}
\title{Experimento de arrastre de conchas}

\begin{document}

\maketitle

\section{Análisis adimensional}

El transporte de sólidos en un fluido turbulento viene gobernado por la relación
entre 3 fuerzas:
\begin{itemize}
\item la fuerza de sustención $F_s$ que tiende a elevar el sólido,
\item la fuerza de la gravedad $F_g$ que tiende a hundirlo,
\item y la fuerza de flotación $F_b$ que tiende a hacerlo flotar.
\end{itemize}

Podemos establecer una relación adimensional que gobernará el tipo de transporte
con la fracción:
\begin{equation}\label{EqForces}
\frac{F_s}{F_g-F_b}
\end{equation}

Para un fluido turbulento de velocidad característica $U_f$ y densidad $\rho_f$,
y para un sólido de densidad $\rho_s$ y tamaño característico $L_s$, la fuerza
de sustentación es proporcional a la presión de ariete por la sección eficaz al
flujo:
\begin{equation}
F_s\;\alpha\;\rho_f\,U_f^2\,L_s^2.
\end{equation}
El peso del sólido será su volumen por su densidad, de modo que:
\begin{equation}
F_g\;\alpha\;\rho_s\,L_s^3\,g.
\end{equation}
Mientras que la fuerza de flotación sera el peso del fluido desalojado, es
decir que:
\begin{equation}
F_b\;\alpha\;\rho_f\,L_s^3\,g.
\end{equation}
Sustituyendo en la ecuación que relaciona las fuerzas (\ref{EqForces}) el tipo
de tranporte se regirá por el número adimensional T:
\begin{equation}
\mathrm{T}\;\alpha\;\frac{F_s}{F_g-F_b}
\;\alpha\;\frac{\rho_f\,U_f^2\,L_s^2}{\rho_s\,L_s^3\,g-\rho_f\,L_s^3\,g}
=\frac{U_f^2}{\left(\frac{\rho_s}{\rho_f}-1\right)\,L_s\,g}.
\end{equation}

\section{Ecuación de transporte}

La ecuaciones que representa el transporte unidimensional promediado en la
sección de solutos y de conchas en una tubería prismática de sección circular se
expresan como ecuaciones típicas de convección-difusión:
\[
\frac{\partial c}{\partial t}+U\,\frac{\partial c}{\partial x}=
K_x\,\frac{\partial c}{\partial x^2},
\]
\begin{equation}
\frac{\partial s}{\partial t}+a\,U\,\frac{\partial s}{\partial x}=
\Gamma\,\frac{\partial s}{\partial x^2},
\end{equation}
con:
\begin{description}
\item[$c$] la densidad volúmetrica de soluto,
\item[$s$] la densidad volumétrica de conchas,
\item[$U$] la velocidad del fluido promediada en la sección,
\item[$a$] un factor adimensional que indica la relación entre la velocidad
media de las conchas frente a la velocidad del fluido,
\item[$K_x$] el coeficiente longitudinal de dispersión del soluto y
\item[$\Gamma$] el coeficiente longitudinal de dispersión de las conchas.
\end{description}

Esta ecuación se resuelve con un esquema numérico explícito de segundo orden TVD
para el término de convección y con un esquema implícito de segundo orden
centrado para el término de difusión.

Para el coeficiente de dispersión de solutos estamos usando una ley basada en
los trabajos de Rutherford:
\begin{equation}
K_x=10\,P\,U\,\sqrt{\frac{f}{2}}
\end{equation}
con $P$ el perímetro de la tubería y $f$ el factor de fricción de
Darcy-Weissbach. Teniendo en cuenta que en flujo turbulento $f\in[0.006,0.06]$ y
que la tubería es $\approx0.1$ m de diámetro y la velocidad de flujo
$U\approx1$~m/s $\Rightarrow$ $K_x\approx0.4$~m$^2$/s.

\section{Resultados de la simulación}

\subsection{Calibración con la cámara cenital final}

Se ha normalizado tanto las densidades medidas con todas las cámaras. La
densidad normalizada de la cámara de inyección es usada como entrada del modelo
de simulación. El tubo se ha simulado con 400 celdas. Las densidades
normalizadas de la cámara cenital final se han usado para calibrar los
coeficientes $a$ y $\Gamma$ minimizando el error cuadrático medio. Los
resultados se presentan en el cuadro~\ref{TabCalibration}.

\begin{table}[ht!]
\centering
\caption{Valores calibrados de los coeficientes $a$ y $\Gamma$.
\label{TabCalibration}}
\begin{tabular}{ccc}
$U$ (m/s) & $a$ (-) & $\Gamma$ (m$^2$/s) \\
\hline
$0.6$ & $0.37$ & $0.11$ \\
$1.8$ & $0.73$ & $0.016$
\end{tabular}
\end{table}

\newpage
\subsection{Resultados de las cámaras cenitales}

\begin{figure}[ht!]
\centering
\includegraphics[width=0.8\textwidth]{v0,6-1/r0,6-1.pdf}
\caption{Medidas normalizadas de densidad larvaria en la cámara frontal en el
punto de inyección, en la cámara cenital final y simuladas en la posición final
para el experimento de velocidad $0.6$~m/s.}
\end{figure}

\begin{figure}[ht!]
\centering
\includegraphics[width=0.8\textwidth]{v1,8-1/r1,8-1.pdf}
\caption{Medidas normalizadas de densidad larvaria en la cámara frontal en el
punto de inyección, en la cámara cenital final y simuladas en la posición final
para el experimento de velocidad $1.8$~m/s.}
\end{figure}

\newpage
\subsection{Resultados de las cámaras frontales}

\begin{figure}[ht!]
\centering
\includegraphics[width=0.8\textwidth]{v0,6-1/f0,6-1.pdf}
\caption{Medidas normalizadas de densidad larvaria en la cámara frontal en el
punto de inyección, en la cámara frontal final y simuladas en la posición final
para el experimento de velocidad $0.6$~m/s.}
\end{figure}

\begin{figure}[ht!]
\centering
\includegraphics[width=0.8\textwidth]{v1,8-1/f1,8-1.pdf}
\caption{Medidas normalizadas de densidad larvaria en la cámara frontal en el
punto de inyección, en la cámara frontal final y simuladas en la posición final
para el experimento de velocidad $1.8$~m/s.}
\end{figure}

\newpage
\subsection{Resultados de los micros más próximos}

\begin{figure}[ht!]
\centering
\includegraphics[width=0.8\textwidth]{v0,6-1/m0,6-1-10.pdf}
\caption{Medidas normalizadas de densidad larvaria en la cámara frontal en el
punto de inyección, en el micro más cercano al final y simuladas en la posición
final para el experimento de velocidad $0.6$~m/s.}
\end{figure}

\begin{figure}[ht!]
\centering
\includegraphics[width=0.8\textwidth]{v1,8-1/m1,8-1-10.pdf}
\caption{Medidas normalizadas de densidad larvaria en la cámara frontal en el
punto de inyección, en el micro más cercano al final y simuladas en la posición
final para el experimento de velocidad $1.8$~m/s.}
\end{figure}

\newpage
\section{Tareas}

\begin{itemize}
\item Calibrar factores $a$ y $\Gamma$ para toda la batería de experimentos.
\item Obtener relaciones $a=a(T)$, $\Gamma=b(T)\,K_x$.
\item Comprobar si se obtienen parecidos valores de $a$ y $b$ con los micros y
	con las cámaras.
\item Obtener alguna ley de distribución vertical de conchas.
\end{itemize}

Es esperable teóricamente:
\begin{figure}[ht!]
\begin{tabular}{cc}
\begin{picture}(160,160)
	\thicklines
	\put(10,10){\line(1,0){50}}
	\put(60,10){\line(0,1){10}}
	\qbezier(60,20)(105,80)(150,80)
	\thinlines
	\put(10,10){\vector(1,0){140}}
	\put(10,10){\vector(0,1){140}}
	\put(10,80){\line(1,0){140}}
	\put(150,0){$U$}
	\put(55,0){$U_c$}
	\put(0,150){$a$}
	\put(0,76){1}
\end{picture}
&
\begin{picture}(160,160)
	\thicklines
	\put(10,10){\line(1,0){50}}
	\put(60,10){\line(0,1){140}}
	\qbezier(60,150)(70,40)(80,40)
	\qbezier(80,40)(125,80)(150,80)
	\thinlines
	\put(10,10){\vector(1,0){140}}
	\put(10,10){\vector(0,1){140}}
	\put(10,80){\line(1,0){140}}
	\put(150,0){$U$}
	\put(55,0){$U_c$}
	\put(0,150){$b$}
	\put(0,77){1}
\end{picture}
\end{tabular}
\end{figure}

\end{document}
