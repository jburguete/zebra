\documentclass[a4paper]{article}

\usepackage{graphicx}
\usepackage[spanish]{babel}

\author{Javier Burguete, Javier Fernández-Pato, Borja Latorre,\\ Nery Zapata}

\date{\today}

\title{Modelo de propagación de mejillón cebra en redes tuberías de riego por
aspersión. Aplicación en la red de la Comunidad de Regantes de Montesusín}

\newcommand{\FIG}[3]
{
	\begin{figure}[ht!]
		\centering
		\includegraphics[width=\textwidth]{#1}
		\caption{#2.\label{#3}}
	\end{figure}
}

\newcommand{\FIGII}[4]
{
	\begin{figure}[ht!]
		\centering
		\includegraphics[width=0.49\textwidth]{#1}
		\includegraphics[width=0.49\textwidth]{#2}
		\caption{#3.\label{#4}}
	\end{figure}
}

\newcommand{\FIGIII}[5]
{
	\begin{figure}[ht!]
		\centering
		\includegraphics[width=0.49\textwidth]{#1}
		\includegraphics[width=0.49\textwidth]{#2}
		\includegraphics[width=0.49\textwidth]{#3}
		\caption{#4.\label{#5}}
	\end{figure}
}

\newcommand{\FIGIV}[6]
{
	\begin{figure}[ht!]
		\centering
		\includegraphics[width=0.49\textwidth]{#1}
		\includegraphics[width=0.49\textwidth]{#2}
		\includegraphics[width=0.49\textwidth]{#3}
		\includegraphics[width=0.49\textwidth]{#4}
		\caption{#5.\label{#6}}
	\end{figure}
}

\newcommand{\FIGV}[7]
{
	\begin{figure}[ht!]
		\centering
		\includegraphics[width=0.49\textwidth]{#1}
		\includegraphics[width=0.49\textwidth]{#2}
		\includegraphics[width=0.49\textwidth]{#3}
		\includegraphics[width=0.49\textwidth]{#4}
		\includegraphics[width=0.49\textwidth]{#5}
		\caption{#6.\label{#7}}
	\end{figure}
}

\begin{document}

\maketitle

\section{Efectos debidos a las larvas}

\subsection{Efecto de la variabilidad en la entrada}

Hipótesis:
\begin{itemize}
\item Red de la Comunidad de Regantes de Montesusín.
\item Tiempo medio de pegado $\approx$ 4 días.
\item Velocidad máxima de pegado 1.2 m/s.
\item No muerte.
\end{itemize}

\FIGIV{2021-mussel-input-constant.eps}{2021-mussel-input-measured.eps}
{2021-mussel-input-uniform.eps}{2021-mussel-input-random.eps}
{Entradas de larvas en función del tiempo. Año 2021}{Fig2021MusselInput}

\FIGIV{2021-mussel-constant.pdf}{2021-mussel-measured.pdf}
{2021-mussel-uniform.pdf}{2021-mussel-random.pdf}
{Infestación con entrada de larvas constante en una ventana de tiempo
(arriba-izquierda), medida (arriba-derecha), uniforme (abajo-izquierda) y
aleatoria (abajo-derecha). Año 2021}{Fig2021MusselRandom}

\FIGIV{2022-mussel-input-constant.eps}{2022-mussel-input-measured.eps}
{2022-mussel-input-uniform.eps}{2022-mussel-input-random.eps}
{Entradas de larvas en función del tiempo. Año 2022}{Fig2022MusselInput}

\FIGIV{2022-mussel-constant.pdf}{2022-mussel-measured.pdf}
{2022-mussel-uniform.pdf}{2022-mussel-random.pdf}
{Infestación con entrada de larvas constante en una ventana de tiempo
(arriba-izquierda), medida (arriba-derecha), uniforme (abajo-izquierda) y
aleatoria (abajo-derecha). Año 2022}{Fig2022MusselRandom}

\clearpage
\subsection{Efecto de la velocidad máxima de pegado}

Hipótesis:
\begin{itemize}
\item Red de la Comunidad de Regantes de Montesusín, año 2021.
\item Entrada de larvas constante.
\item Tiempo medio de pegado $\approx$ 4 días.
\item No muerte.
\end{itemize}

\FIGIII{2021-mussel-velocity-1.0.pdf}{2021-mussel-constant.pdf}
{2021-mussel-velocity-2.0.pdf}{Infestación con velocidad máxima de pegado de
1.0~m/s (arriba-izquierda), 1.2~m/s (arriba-derecha) y 2.0~m/s (abajo). Año
2021}{Fig2021MusselVelocity}

\FIGIII{2022-mussel-velocity-1.0.pdf}{2022-mussel-constant.pdf}
{2022-mussel-velocity-2.0.pdf}{Infestación con velocidad máxima de pegado de
1.0~m/s (arriba-izquierda), 1.2~m/s (arriba-derecha) y 2.0~m/s (abajo). Año
2022}{Fig2022MusselVelocity}

\clearpage
\subsection{Efecto de la probabilidad de pegado}

Hipótesis:
\begin{itemize}
\item Red de la Comunidad de Regantes de Montesusín.
\item Entrada de larvas constante.
\item Velocidad máxima de pegado 1.2 m/s.
\item No muerte.
\end{itemize}

\FIGIII{2021-mussel-cling-1.1.pdf}{2021-mussel-constant.pdf}
{2021-mussel-cling-8.0.pdf}{Infestación con tiempo medio de pegado de
14 días (arriba-izquierda), 4 días (arriba-derecha) y 1 día (abajo). Año 2021}
{Fig2021MusselCling}

\FIGIII{2022-mussel-cling-1.1.pdf}{2022-mussel-constant.pdf}
{2022-mussel-cling-8.0.pdf}{Infestación con tiempo medio de pegado de
14 días (arriba-izquierda), 4 días (arriba-derecha) y 1 día (abajo). Año 2022}
{Fig2022MusselCling}

\clearpage
\subsection{Efecto de la probabilidad de muerte}

Hipótesis:
\begin{itemize}
\item Red de la Comunidad de Regantes de Montesusín.
\item Entrada de larvas constante.
\item Tiempo medio de pegado $\approx$ 4 días.
\item Velocidad máxima de pegado 1.2 m/s.
\item Tasa de muerte de larvas: a los 14 días sobrevive el 10\%.
\item Tasa de muerte de adultos: a los 4 años sobrevive el 10\%.
\end{itemize}

\FIGII{2021-mussel-constant.pdf}{2021-mussel-death.pdf}{Infestación sin muerte
de larvas (izquierda) y con muerte (derecha). Año 2021}{Fig2021MusselDeath}

\FIGII{2022-mussel-constant.pdf}{2022-mussel-death.pdf}{Infestación sin muerte
de larvas (izquierda) y con muerte (derecha). Año 2022}{Fig2022MusselDeath}

\clearpage
\section{Efectos debidos a los venenos}

\subsection{Propagación de cloro en la red de Montesusín}

Hipótesis:
\begin{itemize}
\item Red de la Comunidad de Regantes de Montesusín.
\item Tasa de decaimiento del cloro: 0.36 día$^{-1}$.
\end{itemize}

\FIGIV{2021-chlorine-constant-0.eps}{2021-chlorine-constant-2.eps}
{2021-chlorine-constant-6665.eps}{2021-chlorine-constant-172.eps}
{Concentración de cloro en distintos nodos con entrada constante. Año 2021}
{Fig2021ChlorineConstant}

\FIGIV{2022-chlorine-constant-0.eps}{2022-chlorine-constant-2.eps}
{2022-chlorine-constant-68.eps}{2022-chlorine-constant-172.eps}
{Concentración de cloro en distintos nodos con entrada constante. Año 2022}
{Fig2022ChlorineConstant}

\begin{table}[ht!]
	\centering
	\caption{Lista de nodos sin cloro.\label{TabWithoutChlorine}}
	\begin{tabular}{cc}
		2021 & 2022 \\
		\hline
                3004 & 3004 \\
		3008 & 3008 \\
		3020 & 3020 \\
		3019 & 3019 \\
		172 & 172 \\
		61 & 61 \\
		3012 & 3012 \\
		2003 & 2003 \\
		3013 & 3013 \\
		3024 & 3024 \\
		6635 & 6635 \\
		3661 & 3661
	\end{tabular}	
\end{table}

\FIGV{2021-chlorine-day-0.eps}{2021-chlorine-day-2.eps}
{2021-chlorine-day-65.eps}{2021-chlorine-day-6665.eps}
{2021-chlorine-day-172.eps}
{Concentración de cloro en distintos nodos con entrada con una ventana horaria
diaria. Año 2021}{Fig2021ChlorineDay}

\FIGV{2022-chlorine-day-0.eps}{2022-chlorine-day-2.eps}
{2022-chlorine-day-6634.eps}{2022-chlorine-day-68.eps}
{2022-chlorine-day-172.eps}
{Concentración de cloro en distintos nodos con entrada con una ventana horaria
diaria. Año 2022}{Fig2022ChlorineDay}

\FIGV{2021-chlorine-week-0.eps}{2021-chlorine-week-2.eps}
{2021-chlorine-week-68.eps}{2021-chlorine-week-123.eps}
{2021-chlorine-week-172.eps}
{Concentración de cloro en distintos nodos con entrada con una ventana horaria
semanal. Año 2021}{Fig2021ChlorineWeek}

\FIGV{2022-chlorine-week-0.eps}{2022-chlorine-week-2.eps}
{2022-chlorine-week-6634.eps}{2022-chlorine-week-139.eps}
{2022-chlorine-week-172.eps}
{Concentración de cloro en distintos nodos con entrada con una ventana horaria
semanal. Año 2022}{Fig2022ChlorineWeek}

\end{document}
