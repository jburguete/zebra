\documentclass[pdr]{beamer}
\mode<presentation>{\usetheme{Marburg}}
\usepackage[utf8]{inputenc}
\usepackage[spanish]{babel}
\usepackage{pstricks}
\usepackage{multido}
\usepackage{listings,multicol}
%\usepackage{minted}
\usepackage{pstricks}
\usepackage{pst-plot}

\newcommand{\PSPICTURE}[5]
{
	\begin{pspicture}(#1,#2)(#3,#4)
		#5
	\end{pspicture}
}

\newcommand{\FIGURE}[2]
{
	\begin{center}
		\includegraphics[width=#1]{#2}
	\end{center}
}

\title[Análisis de los tratamientos químicos para el control del mejillón cebra
	en redes de riego presurizadas]
{Análisis del movimiento de la materia activa de los tratamientos químicos para
el control del mejillón cebra en las redes de riego presurizadas}
\author[{\bfseries Javier Burguete}\\Estación Experimental de Aula Dei / CSIC]
{{\bfseries Javier Burguete Tolosa}\\
	Suelo y Agua. Estación Experimental de Aula Dei / CSIC}
\date{1 de marzo de 2024}

\begin{document}

\begin{frame}
	\begin{titlepage}
		\centering
		\FIGURE{0.8\textwidth}{Logos.pdf}
	\end{titlepage}
\end{frame}

\begin{frame}{Índice}
	\tableofcontents
\end{frame}

\section{Introducción}

\subsection{Venenos para el mejillón cebra}

\begin{frame}{Venenos para el mejillón cebra}
	\begin{itemize}
		\item Agua oxigenada
		\item Cloro
		\item Sustancias muy oxidantes: "queman" la materia orgánica
		\item Reaccionan con las sustancias disueltas en el agua y con
			las paredes de la tuberías
		\item Decaen rápidamente en el tiempo
		\item Son corrosivas
		\item El cloro puede generar subproductos tóxicos
		\item Las larvas son más sensibles que los adultos
		\item Una concentración de 0.5 ppm de cloro (similar a los
			tratamientos de agua potable) mata todas las larvas en
			una hora
		\item A las concentraciones que matan las larvas, el riego no
			parece producir daños graves en plantas
	\end{itemize}
\end{frame}

\subsection{Zona de estudio}

\begin{frame}{Zona de estudio}
	\framesubtitle{Ramal de la Comunidad de Regantes de Montesusín}
\end{frame}

\section{Herramientas de cálculo}

\subsection{Modelo de movimiento de solutos}

\begin{frame}{Modelo de movimiento de solutos}
	\begin{itemize}
		\item Requiere:
		\begin{itemize}
			\item Datos del proyecto de la red (EPANET)
			\begin{itemize}
				\item Longitud, diámetro y material de cada
					tubería
			\end{itemize}
			\item Datos temporales de caudal de cada tubería
			\begin{itemize}
				\item Se calculan simulando la red con EPANET a
					partir de los datos de demanda en cada
					hidrante (Telecontrol)
			\end{itemize}
			\item Calibración de coeficientes de decaimiento
		\end{itemize}
		\item El modelo es muy preciso calculando transporte y
			dispersión de solutos
		\item Hay más incertidumbre en el decaimiento 
	\end{itemize}
\end{frame}

\subsection{Modelo de movimiento de larvas}

\begin{frame}{Modelo de movimiento de larvas}
	\begin{itemize}
		\item Las larvas se mueven y dispersan como cualquier soluto
		\item Requiere:
		\begin{itemize}
			\item Datos del proyecto de la red (EPANET)
			\begin{itemize}
				\item Longitud, diámetro y material de cada
					tubería
			\end{itemize}
			\item Datos temporales de caudal de cada tubería
			\begin{itemize}
				\item Se calculan simulando la red con EPANET a
					partir de los datos de demanda en cada
					hidrante (Telecontrol)
			\end{itemize}
			\item Calibración de coeficientes de decaimiento
		\end{itemize}
		\item El modelo es muy preciso calculando transporte y
			dispersión de larvas
		\item Sin embargo, la cantidad de larvas que entran es muy
			variable
	\end{itemize}
\end{frame}

\subsection{Modelo de crecimiento de mejillones}

\begin{frame}{Modelo de crecimiento de mejillones}
\end{frame}

\section{Efecto de diferentes tratamientos químicos}

\subsection{Tratamiento continuo}

\begin{frame}{Tratamiento continuo}
\end{frame}

\subsection{Tratamientos discontinuos}

\section{Conclusiones}

\begin{frame}{Conclusiones}
	\begin{itemize}
		\item Disponemos de 2 sustancias efectivas para eliminar las
			larvas de mejillón cebra: cloro y agua oxigenada
		\item Ambos son sustancias peligrosas
		\item Los subproductos del agua oxigenada son menos peligrosos
			pero hay que echar mayor cantidad (¿más caro?)
		\item El tratamiento en continuo es el más seguro para evitar la
			presencia del mejillón
		\item No parece que estos tratamientos causen mermas
			significas de producción en los cultivos
		\item Los costes parecen asumibles
		\item Hemos diseñado unas herramientas que podrían ayudar a
			reducir costes e impacto de los tratamientos
	\end{itemize}
\end{frame}

\section{Trabajo futuro}

\begin{frame}{Trabajo futuro}
	\begin{itemize}
		\item Determinar coeficientes de decaimiento de cloro en una red
			de riego
		\item Determinar coeficientes de decaimiento del agua oxigenada
		\item Refinar el modelo de crecimiento de mejillones
		\item Diseñar un modelo de arrastre de conchas
		\item Diseñar una interfaz y un manual para los modelos
		\item Probar los modelos con diferentes tratamientos químicos
		\item Usar los modelos en más zonas
	\end{itemize}
\end{frame}

\end{document}
