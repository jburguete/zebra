\documentclass[pdr]{beamer}
\mode<presentation>{\usetheme{Marburg}}
\usepackage[utf8]{inputenc}
\usepackage[spanish]{babel}
\usepackage{pstricks}
\usepackage{multido}
\usepackage{listings,multicol}
%\usepackage{minted}
\usepackage{pstricks}
\usepackage{pst-plot}

\newcommand{\PSPICTURE}[5]
{
	\begin{pspicture}(#1,#2)(#3,#4)
		#5
	\end{pspicture}
}

\newcommand{\FIGURE}[2]
{
	\begin{center}
		\includegraphics[width=#1]{#2}
	\end{center}
}

\newcommand{\FIGII}[2]
{
	\begin{center}
		\includegraphics[width=0.45\textwidth]{#1}
		\includegraphics[width=0.45\textwidth]{#2}
	\end{center}
}

\newcommand{\FIGIIB}[4]
{
	\begin{figure}[ht!]
		\centering
		\begin{tabular}{cc}
			#2&#4\\
			\includegraphics[width=0.45\textwidth]{#1}&
			\includegraphics[width=0.45\textwidth]{#3}
		\end{tabular}
	\end{figure}
}

\newcommand{\FIGIIC}[5]
{
	\begin{figure}[ht!]
		\centering
		\begin{tabular}{cc}
			#2&#4\\
			&#5\\
			\includegraphics[width=0.45\textwidth]{#1}&
			\includegraphics[width=0.45\textwidth]{#3}
		\end{tabular}
	\end{figure}
}

\newcommand{\FIGIV}[4]
{
	\begin{center}
		\includegraphics[width=0.49\textwidth]{#1}
		\includegraphics[width=0.49\textwidth]{#2}
		\includegraphics[width=0.49\textwidth]{#3}
		\includegraphics[width=0.49\textwidth]{#4}
	\end{center}
}

\newcommand{\FIGIIIB}[5]
{
	\begin{figure}[ht!]
		\centering
		\begin{tabular}{cc}
			#2&#3\\
			\includegraphics[width=0.45\textwidth]{#1}&\\
			\multicolumn{2}{c}{#5}\\
			\multicolumn{2}{c}
			{\includegraphics[width=0.45\textwidth]{#4}}
		\end{tabular}
	\end{figure}
}

\newcommand{\FIGIVB}[8]
{
	\begin{figure}[ht!]
		\centering
		\begin{tabular}{cc}
			#2&#4\\
			\includegraphics[width=0.45\textwidth]{#1}&
			\includegraphics[width=0.45\textwidth]{#3}\\
			#6&#8\\
			\includegraphics[width=0.45\textwidth]{#5}&
			\includegraphics[width=0.45\textwidth]{#7}
		\end{tabular}
	\end{figure}
}

\title[Análisis de los tratamientos químicos para el control del mejillón cebra
	en redes de riego presurizadas]
{Análisis del movimiento de la materia activa de los tratamientos químicos para
el control del mejillón cebra en las redes de riego presurizadas}
\author[{\bfseries Javier Burguete}\\Grupo Rama, EEAD / CSIC]
{{\bfseries Javier Burguete}, Nery Zapata, Javier Fernández, Borja Latorre,
	Enrique Playán\\
	Grupo Rama. Estación Experimental de Aula Dei / CSIC}
\date{1 de marzo de 2024}

\begin{document}

\begin{frame}
	\begin{titlepage}
		\centering
		\FIGURE{0.9\textwidth}{Logos.pdf}
	\end{titlepage}
\end{frame}

\begin{frame}{Índice}
	\tableofcontents
\end{frame}

\section{Introducción}

\subsection{Tratamientos para el mejillón cebra}

\begin{frame}{Tratamientos para el mejillón cebra}
	\begin{itemize}
		\item 2 venenos: agua oxigenada y cloro
		\item Sustancias muy oxidantes: "queman" la materia orgánica
		\item Reaccionan con las sustancias disueltas en el agua y con
			las paredes de la tuberías
		\item Decaen rápidamente en el tiempo
	\end{itemize}
\end{frame}

\begin{frame}{Tratamientos para el mejillón cebra}
	\framesubtitle{Decaimiento del cloro}
	\FIGURE{\textwidth}{chlorine-decay.eps}
\end{frame}

\begin{frame}{Tratamientos para el mejillón cebra}
	\begin{itemize}
		\item 2 venenos: agua oxigenada y cloro
		\item Sustancias muy oxidantes: "queman" la materia orgánica
		\item Reaccionan con las sustancias disueltas en el agua y con
			las paredes de la tuberías
		\item Decaen rápidamente en el tiempo
		\item Son sustancias muy peligrosas
		\item Son corrosivas
		\item El cloro puede generar subproductos tóxicos
		\item Las larvas son más sensibles que los adultos
		\item Una concentración de 0.5 ppm de cloro libre (similar a los
			tratamientos de agua potable) mata todas las larvas en
			una hora
		\item A las concentraciones que matan las larvas, el riego no
			parece producir daños graves en plantas
	\end{itemize}
\end{frame}

\subsection{Zona de estudio}

\begin{frame}{Zona de estudio}
	\framesubtitle{Ramal de la Comunidad de Regantes de Montesusín}
	\FIGURE{0.65\textwidth}{Montesusín.png}
\end{frame}

\section{Herramientas de cálculo}

\subsection{Modelo de movimiento de solutos}

\begin{frame}{Modelo de movimiento de solutos}
	\begin{itemize}
		\item Requiere:
		\begin{itemize}
			\item Datos del proyecto de la red (EPANET)
			\begin{itemize}
				\item Longitud, diámetro y material de cada
					tubería
			\end{itemize}
			\item Datos temporales de caudal de cada tubería
			\begin{itemize}
				\item Se calculan simulando la red con EPANET a
					partir de los datos de demanda en cada
					hidrante (Telecontrol)
			\end{itemize}
			\item Calibración de coeficientes de decaimiento
		\end{itemize}
		\item El modelo es muy preciso calculando transporte y
			dispersión de solutos
		\item Hay más incertidumbre en el decaimiento 
	\end{itemize}
\end{frame}

\begin{frame}{Modelo de movimiento de solutos}
	\FIGII{24-2022-chlorine-constant-0.eps}{24-2022-chlorine-constant-2.eps}
\end{frame}

\subsection{Modelo de movimiento de larvas}

\begin{frame}{Modelo de movimiento de larvas}
	\begin{itemize}
		\item Las larvas se mueven y dispersan como cualquier soluto
		\item Requiere:
		\begin{itemize}
			\item Datos del proyecto de la red (EPANET)
			\begin{itemize}
				\item Longitud, diámetro y material de cada
					tubería
			\end{itemize}
			\item Datos temporales de caudal de cada tubería
			\begin{itemize}
				\item Se calculan simulando la red con EPANET a
					partir de los datos de demanda en cada
					hidrante (Telecontrol)
			\end{itemize}
		\end{itemize}
		\item El modelo es muy preciso calculando transporte y
			dispersión de larvas
		\item Sin embargo, la cantidad de larvas que entran es muy
			variable
	\end{itemize}
\end{frame}

\begin{frame}{Modelo de movimiento de larvas}
	\framesubtitle{Concentración de larvas en una arqueta de la red de
		Montesusín}
	\FIGURE{\textwidth}{mussel-input-measured.eps}
\end{frame}

\subsection{Modelo de crecimiento de mejillones}

\begin{frame}{Modelo de pegado y crecimiento de mejillones}
	\begin{itemize}
		\item Las larvas se pegan aleatoriamente si la velocidad del
			agua es $<$ 1.2 m/s
		\item Las larvas van creciendo según su edad
		\item Hay una tasa de mortalidad
	\end{itemize}
\end{frame}

\begin{frame}{Modelo de pegado y crecimiento de mejillones}
	\framesubtitle{Entrada de larvas (Montesusín 2021)}
	\FIGIV{2021-mussel-input-constant.eps}{2021-mussel-input-measured.eps}
	{2021-mussel-input-uniform.eps}{2021-mussel-input-random.eps}
\end{frame}

\begin{frame}{Modelo de pegado y crecimiento de mejillones}
	\framesubtitle{Entrada de larvas (Montesusín 2021)}
	\FIGIVB{2021-mussel-constant.pdf}{$+0\%$, R=$1$}
	{2021-mussel-measured.pdf}{$-41.7\%$, R=$0.977$}
	{2021-mussel-uniform.pdf}{$-28.4\%$, R=$0.976$}
	{2021-mussel-random.pdf}{$-9.8\%$, R=$0.990$}
\end{frame}

\begin{frame}{Modelo de pegado y crecimiento de mejillones}
	\framesubtitle{Entrada de larvas (Montesusín 2022)}
	\FIGIVB{2022-mussel-constant.pdf}{$+0\%$, R=$1$}
	{2022-mussel-measured.pdf}{$+24.8\%$, R=$0.976$}
	{2022-mussel-uniform.pdf}{$-21.8\%$, R=$0.957$}
	{2022-mussel-random.pdf}{$-28.3\%$, R=$0.988$}
\end{frame}

\section{Tratamientos químicos}

\subsection{Tratamiento continuo}

\begin{frame}{Tratamiento continuo}
	\begin{itemize}
		\item Consiste en inyectar continuamente en cabecera
			concentraciones de veneno suficiente para matar a las
			larvas
		\item Al matar de raíz las larvas no se crían adultos
		\item Teniendo en cuenta el decaimiento del cloro, bastaría con
			inyectar en cabecera 1 ppm de cloro libre
		\item Si en una campaña se riega del orden de 6000-8000 m$^3$/ha
			con 1 ppm de cloro $\Rightarrow$
			6-8 kg/ha de cloro $\Rightarrow$
			40-50 l/ha de lejía al 15\% $\Rightarrow$
			50 €/ha
		\item Se necesita inyectar ininterrumpidamente dosis
			proporcionales al caudal que circula por la tubería
		\item Es el tratamiento más seguro para evitar la presencia del
			mejillón
	\end{itemize}
\end{frame}

\begin{frame}{Tratamiento continuo}
	\begin{figure}[ht!]
		\centering
		\begin{tabular}{cc}
			Sin tratamiento & Tratamiento continuo (-100\% M)\\
			\includegraphics[width=0.45\textwidth]
			{2022-mussel-constant.pdf}
		\end{tabular}
	\end{figure}
\end{frame}

\subsection{Tratamiento de choque}

\begin{frame}{Tratamiento de choque}
	\begin{itemize}
		\item Consiste en inyectar concentraciones muy altas de veneno
			en momentos puntuales
		\item Mata tanto larvas como adultos
		\item Aunque la concentración inyectada es muy alta, al
			producirse en pequeños lapsos de tiempo, la dosis total
			puede ser mucho menor
		\item Los vertidos se localizan en hidrantes (no afectan al
			cultivo)
		\item Podría ahorrarse dinero y además reducirse el impacto
			ambiental 
		\item Puede se difícil que el veneno se distribuya por toda la
			red con la concentración suficiente 
		\item En redes amplias pueden necesitarse varios puntos de
			inyección
	\end{itemize}
\end{frame}

\subsection{Tratamientos discontinuos}

\begin{frame}{Tratamientos discontinuos}
	\begin{itemize}
		\item Consisten en inyectar a intervalos de tiempo en cabecera
			concentraciones de veneno suficiente para matar a las
			larvas
		\item Se necesita inyectar dosis proporcionales al caudal que
			circula por la tubería
	\end{itemize}
\end{frame}

\begin{frame}{Tratamientos discontinuos}
	\framesubtitle{Consumos horarios}
	\FIGURE{\textwidth}{Consumos.pdf}
\end{frame}

\begin{frame}{Tratamientos discontinuos}
	\FIGIIC{2022-mussel-constant.pdf}{Sin tratamiento}
	{24-2021-chlorine-day.pdf}
	{Tratamiento de 8 a 20}{$\approx$ -70\% Cl, -97,1\% M}
\end{frame}

\begin{frame}{Tratamientos discontinuos}
	\FIGIIC{2022-mussel-constant.pdf}{Sin tratamiento}
	{24-2021-chlorine-dayb.pdf}
	{Tratamiento de 20 a 8}{$\approx$ -30\% Cl, -99,3\% M}
\end{frame}

\begin{frame}{Tratamientos discontinuos}
	\FIGIIC{2022-mussel-constant.pdf}{Sin tratamiento}
	{24-2021-chlorine-hour.pdf}
	{Tratamiento hora sí hora no}{$\approx$ -50\% Cl, -98,9\% M}
\end{frame}

\begin{frame}{Tratamientos discontinuos}
	\begin{itemize}
		\item Consisten en inyectar a intervalos de tiempo en cabecera
			concentraciones de veneno suficiente para matar a las
			larvas
		\item Se necesita inyectar dosis proporcionales al caudal que
			circula por la tubería
		\item Podríamos ahorrar costes e impactos
		\item Tenemos que analizar la eficacia: existe el peligro de que
			larvas vivas alcancen zonas en las que el veneno no
			llegue nunca con la concentración suficiente para
			matarlas
	\end{itemize}
\end{frame}

\section{Conclusiones}

\begin{frame}{Conclusiones}
	\begin{itemize}
		\item Disponemos de 2 sustancias efectivas para eliminar las
			larvas de mejillón cebra: cloro y agua oxigenada
		\item Ambos son sustancias peligrosas
		\item Los subproductos del agua oxigenada son menos peligrosos
			pero hay que echar mayor cantidad (más caro)
		\item El tratamiento en continuo es el más seguro para evitar la
			presencia del mejillón
		\item No parece que estos tratamientos causen mermas
			significativas de producción en los cultivos
		\item Los costes y el impacto ambiental podrían ser excesivos
		\item Estamos diseñando herramientas que podrían ayudar a
			reducir costes e impacto de los tratamientos
	\end{itemize}
\end{frame}

\section{Tareas}

\begin{frame}{Tareas}
	\begin{itemize}
		\item Determinar coeficientes de decaimiento de cloro en una red
			de riego
		\item Determinar coeficientes de decaimiento del agua oxigenada
		\item Refinar el modelo de crecimiento de mejillones
		\item Diseñar un modelo de arrastre de conchas
		\item Usar los modelos en más zonas y simular la eficacia de los
			diferentes tratamientos químicos
		\item Integrar los modelos en la interfaz gráfica
	\end{itemize}
\end{frame}

\end{document}
